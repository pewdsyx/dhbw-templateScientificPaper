\chapter*{Kurzfassung}
\addcontentsline{toc}{chapter}{Abstract}
\begingroup
\begin{table}[h!]
\setlength\tabcolsep{0pt}
\begin{tabular}{p{3.7cm}p{11.7cm}}
Titel: & \DerTitelDerArbeit \\
Verfasser/-in: & \DerAutorDerArbeit \\
Kurs: & \DieKursbezeichnung \\
Ausbildungsstätte: & \DerNameDerFirma\\
\end{tabular}
\end{table}
\endgroup

%Überlege, ob ich den Header brauche mit den ganzen Infos? --> JA.
%Hier können Sie die Kurzfassung der Arbeit schreiben. 
Die \ac{SV} hat sich zum Ziel gesetzt, ihre Versicherungsprozesse und den Kontakt mit den Kunden durch digitale Kanäle zu verbessern. Die Digitalisierung ist ein wichtiger Bestandteil der \ac{SV}-Strategie: So ist sie Mitbegründerin der \enquote{id-fabrik}\footnote{Ein Start-up, das federführend Innovationen im Bereich der S-Finanzgruppe erzeugt.} und Mitglied im \enquote{InsurLab Germany}\autocite[vgl.][S.\,30]{sv_sparkassenversicherung_sv_2019}. Die Digitalisierungsanforderungen sollen durch eine Container-Plattform umgesetzt werden. Ziel der Arbeit ist, eine Betrachtung der technischen Umsetzbarkeit eines Container-\enquote{Deployments} durchzuführen und einen Beispielprozess zu implementieren. Des Weiteren sollen die wirtschaftlichen und sicherheitsrelevanten Aspekte dieses Prozesses bzw. der Container-Anwendungen betrachtet werden. Methodisch werden Soll-Ist-Vergleiche, Anforderungsanalyse und Pseudocode verwendet. 
\par
Die Forschungsfrage eins beschäftigt sich mit der Frage, wie Container-Anwendungen den Prozess des automatisierten \enquote{Deployments} unterstützen können. Diese Frage beschäftigt sich mit der Anforderungserhebung und der Implementierung eines angepassten Prozesses. Abschließend ist ein generischer Prozess entstanden, der Container-Anwendungen auf eine Orchestrierungsplattform verteilt. 
\par
Die Forschungsfrage zwei analysiert, welchen wirtschaftlichen Vorteil die Container-Anwendungen für ein Unternehmen bieten. Dabei wird ein \enquote{Business case} aufgebaut, der die Investition in Container-Anwendungen analysiert und einordnet. Die Frage schließt mit dem Ergebnis ab, dass die formal korrekte Betrachtung eines \enquote{Business case} meist in der Unternehmensrealität nicht vollständig umgesetzt wird.
\par
Die Forschungsfrage drei beschäftigt sich mit der Analyse der Sicherheit des neuen Prozesses. Dabei wird sich des IT-Grundschutz-Kompendiums des \ac{BSI} bedient, das vorgefertigte Sicherheitsbausteine für IT-Komponenten beschreibt. Diese werden in einem Soll-Ist-Vergleich mit den bereits implementierten Sicherheitsanforderungen der \ac{SVI}\footnote{Dies ist die IT-Dienstleisterin der \ac{SV}.} verglichen. 
