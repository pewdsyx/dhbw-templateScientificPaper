% !TEX root =  master.tex

\clearpage
\chapter*{Ehrenwörtliche Erklärung}

% Wird die folgende Zeile auskommentiert, erscheint die ehrenwörtliche
% Erklärung im Inhaltsverzeichnis.
\addcontentsline{toc}{chapter}{Ehrenwörtliche Erklärung}

Ich versichere hiermit, dass ich die vorliegende Arbeit mit dem Thema: \textit{\DerTitelDerArbeit} selbstständig verfasst und keine anderen als die angegebenen Quellen und Hilfsmittel benutzt habe. Ich versichere zudem, dass die eingereichte elektronische Fassung mit der gedruckten Fassung übereinstimmt.

%\vspace{3cm}
%\noindent\rule{5cm}{.4pt}\hfill\rule{5cm}{.4pt}\par
%Ort, Datum \hfill \DerAutorDerArbeit

\vspace{3cm}
\noindent Mannheim, 08.05.2020 \hfill\rule{5cm}{.4pt}\par
\noindent\hfill \DerAutorDerArbeit
