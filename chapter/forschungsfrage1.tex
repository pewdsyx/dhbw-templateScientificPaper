\chapter[Forschungsfrage 1]{Wie können Container-Anwendungen den Prozess des automatisierten \enquote{Deployments} unterstützen?} \label{ff1}
Dieses Kapitel ...

\section{Grundlagen: Definieren der Begrifflichkeiten zur Forschungsfrage eins}
Dieses Teilkapitel soll grundlegende Begrifflichkeiten, die im weiteren Verlauf dieser Arbeit verwendet werden, definieren, um so eine einheitliche Terminologie der Begriffe zu entwickeln. Dadurch wird ein gemeinsames Verständnis erzeugt.

\subsection{Methodik der Anforderungsanalyse}
Die Anforderungsanalyse leitet sich aus dem thematischen Komplex des \enquote{Requirements-Engineering} ab, die verschiedene Bedeutungsvarianten besitzt -- dabei \enquote{[...] steht [es] einmal für alle konkreten Aktivitäten am Beginn einer Systementwicklung, die auf eine Präzisierung der Problemstellung abzielen. Ebenso steht es aber auch für eine ganze Teildisziplin im Grenzbereich zwischen Systems-Engineering, Informatik und Anwendungswissenschaften.}\autocite[][S.19]{partsch_requirements-engineering_2010} Diese Analyse soll, laut der herrschenden Meinung der Wissenschaft, am Anfang jeder Systementwicklung stehen, um so bestimmte Vorgehensweise anzuwenden. Dabei entstehen, wenn der später weiter definierte Prozess verfolgt wird, viele systematisch verbundene Dokumente, die Anforderungen enthalten. So ist jede Anforderung wieder ein Cluster von kleineren Anforderungen, die miteinander verbunden sind. Diese werden durch den IEEE-Standard 1220 definiert als \enquote{a statement that identifies a product or process operational, functional, or design characteristic or constraint, which is unambiguous, testable or measurable, and necessary for product or process acceptability (by consumers or internal quality assurance guidelines).}\autocite[][S.9]{IEEE1220-2005SystemsEng} Dieser Standard legt mit höchster Priorität den Fokus auf die Formulierung einer Anforderung als elementar wichtig für das Produkt bzw. für das Erreichen der Akzeptanz des Produktes. Ziel der Analyse ist es, funktionale und nicht-funktionale Anforderungen zu identifizieren und diese testbar zu dokumentieren. Funktionale Anforderungen definieren genau, was ein System später erfüllen muss, sie ergeben sich aus der Fragestellung \enquote{Was tut das System?/Was soll es aufgrund der Aufgabenstellung können?}\autocite[][S.27]{partsch_requirements-engineering_2010} Nicht-funktionale Anforderungen konkretisieren die Qualitätsansprüche an das System, die Forderung an das zu implementierende System als Ganzes, sowie Randbedingungen, die aus Projekt-/Prozess-/Unternehmensbedingungen resultieren können.\autocite[vgl.][S.27-29]{partsch_requirements-engineering_2010}

\begin{figure}[H]
	\centering
	\includegraphics[scale=0.38]{img/levels-of-requirements-engineering.pdf}
	\caption{Entwicklungsprozess der Anforderungen}
	{\footnotesize \cite[Quelle: in Anlehnung an ][S.28]{hull_requirements_2011}}
	\label{abb:entwAnforderung}
	%		{\scriptsize \textit{Alle Rechte, einschließlich der Vervielfältigung, Veröffentlichung, Bearbeitung und Übersetzung bleiben der SV Informatik GmbH vorbehalten.}}
\end{figure}

Das \enquote{statement of needs} ist der Startpunkt für die Entwicklung einer Anforderung die am Ende des Prozesses, der in Abbildung \vref{abb:entwAnforderung} dargestellt ist, präzise dokumentiert sein wird. Dieses ist am Anfang immer ein Ausdruck eines Anspruchs oder Wunsches an das zu entwerfende System; dabei bildet das \enquote{statement} und die \enquote{stakeholder requirements} die \enquote{problem domain}. Diese definiert grundständige Methodik, wie auch eine nicht-technische Herangehensweise, die auf die Projektbeteiligten (\enquote{stakeholder}) angepasst ist. Nachfolgend werden die Projektbeteiligen als \enquote{stakeholder} bezeichnen, dabei ist die Rolle beschrieben als \enquote{(Stakeholder) sind Personen oder Organisationen, die ein potenzielles Interesse an einem zukünftigen System haben und somit in der Regel auch Anforderungen an das System stellen.}\autocite[][S.8]{partsch_requirements-engineering_2010} Später definiert die \enquote{problem domain} den Zweck des Systems -- dadurch ist bei der Ermittlung der Anforderungen die Frage \enquote{Was ist der Zweck des Systems?} anstelle \enquote{Was soll das System ihrer Meinung nach tun?}. Dies soll die \enquote{stakeholder} extrinsisch motivieren über den Zweck des zu entwerfenden Systems und nicht über einen möglichen Lösungsweg (das Wie) nachzudenken. Durch diesen Ansatz folgen Antworten nach dem Muster \enquote{Ich möchte etwas tun können ...} -- wissenschaftlich bzw. literarisch betrachtet sind diese Form der Anforderungen als \enquote{capability requirement(s)}\autocite[vgl.][S.94]{hull_requirements_2011} bekannt. Sie stellen die wichtigsten Erkenntnisse in der \enquote{problem domain} dar. Nun wird im weiteren Verlauf ein Modell konstruiert, das den Projektbeteiligten, den \enquote{stakeholder}, präsentiert wird. Dies unterliegt der Einschränkung, dass es jede/jedem Projektbeteiligte/n versteht. Denn sie validieren das konstruiert Modell in jedem weiteren Schritt, der in Abbildung \vref{abb:entwAnforderung}, ersichtlich ist. Die Anforderungen an das Modell sind quantitativ gering: es muss nicht-technisch sein und es muss geeignet sein die Anforderungen an das Systems abzubilden. Eine solche Darstellung ist dann geeignet, wenn sie den gewünschten Zweck an das System abbildet, das heißt, dass sie keine technischen Details zeigt, sondern einen Überblick bietet. Ein \enquote{use scenario}\autocite[vgl.][S.94]{hull_requirements_2011} wird meist verwendet, da es sich eignet menschliche Aktionen bzw. Ziele darzustellen. Abschließend müssen die \enquote{stakeholder}-Anforderungen folgende Kriterien erfüllen: 

\begin{itemize}
	\item kurz und prägnant formulierte Beschreibung, jedoch einfach zu verstehen und
	\item gleichzeitig sollten sie nicht-technisch aber realistisch formuliert sein.
\end{itemize}
 
 Die \enquote{solutions domain}, die auf Abbildung \vref{abb:entwAnforderung} zu sehen ist, ist die Nachfolgerin von der \enquote{problem domain}. Der Hauptunterschied zwischen den beiden Bereichen ist, dass die \enquote{solution domain} idealtypisch qualitativ hochwertig beschriebene Anforderungen als \enquote{Input} bekommt. Dazu konträr erhält die \enquote{problem domain} vage formulierte Wunschliste oder einem nicht klar definierten Ziel als initialen \enquote{Input}. Ausgehend von der Aussage von E. Hull, \enquote{in an ideal world, all the requirements would be clearly articulated, individual test able requirements}\autocite[][S.115]{hull_requirements_2011}, ist zu deduzieren, dass viele Ebenen zu erforschen gibt, um dieser Aufforderung zu entsprechen. So muss iterativ in jeder Ebene eine neue Analyse des \enquote{Inputs} erfolgen, um einen Ausgangspunkt für das weitere Vorgehen zu initialisieren. Die Komplexität diese Ebenen ist anhängig von dem Grad der Innovation sowie vom Kontext des zu entwickelnden Systems. Jede Entscheidung während des Prozess kann mögliche Entscheidungspfade in einer anderen Ebene verhindern. Ziel des Prozesses ist es, ein Anforderungsdokument/-katalog zu entwerfen, das laut der gesichteten Literatur in verschiedenen Repräsentationen vorliegen kann. Dennoch sollten primäre Bestandteile, wie die Rahmenbedingungen, die Projektbeteiligten, die Projektaspekte und die funktionale/nicht-funktionale Anforderungen, enthalten sein. Ein Beispiel dieses Katalogs ist im Anhang \vref{appendixAnforderung} zur Ansicht enthalten.
 
\subsection{\ac{Cloud-C}}
\ac{Cloud-C} ist ein neuartiger und disruptiver Ansatz in der Informationstechnologie, der seit mehreren Jahren Führungskräfte und IT-Abteilungen beschäftigt. Dieser Ansatz verspricht die Lösung für sämtliche Herausforderungen der Kapazitäts- und Leistungsengpässe moderner IT-Infrastruktur zu sein.\autocite[vgl.][S.4]{reinheimer_cloud_2018} Auch diskutiert die Bevölkerung stark und meist auch sehr kontrovers über dieses Thema -- Themen wie Datenschutz und Privatsphäre; Risiko eines Datendiebstahls und die rechtlichen Fragen sind auch nach 20 Jahren Diskussion immer noch allgegenwärtig. Ein Grund dafür ist die hohe Dynamik dieser Technologie, sowie die ständigen Weiterentwicklung, die von großen Unternehmen, wie \textsc{Microsoft}, \textsc{Google}, \textsc{Amazon} und \textsc{IBM}, voran getrieben werden. Momentan haben \textsc{Microsoft} und \textsc{Amazon} die meisten Marktanteile am Umsatz im Bereich des \ac{Cloud-C}.\footnote{siehe dazu Abbildung \vref{abb:marktanteileCC19}} Des weiteren prognostiziert \cite{gartner_cloud_2019} einen exponentiell wachsenden weltweiten Umsatz bis 2022 auf ungefähr 354,6 Milliarden US-Dollar. Damit würde dieser in den nächsten zwei Jahren um circa 100 Milliarden US-Dollar steigen. Für eine ausführliche Umsatzprognose ist auf die Abbildung \vref{abb:umsatzprognoseCC} zu verweisen. Diese verdeutlicht auch, dass in den folgenden Jahren nach 2022 weiterhin mit einer exponentiellen Umsatzsteigerung zu rechnen ist, wenn das mathematische Modell der exponentiellen Regression weiterhin bestand hat. \par
Historisch betrachtet leitet sich \ac{Cloud-C} an verschiedenen Konzepten anderer \enquote{Comput-ing}-Bereiche und Architekturmustern ab: So spielte zur Entwicklung des heutigen Verständnis \enquote{Utility Computing}, \enquote{Service Orientation} und \enquote{Grid Computing} eine große Rolle.\autocite[vgl.][S.3-5]{hill_guide_2013} John McCarthy hat in den 1960er-Jahren das erste Konzept im Bereich des \enquote{Utility Computing} entwickelt.\autocite[vgl.][]{mccarthy_reminiscences_1983} Später wurde es durch Douglas Parkhill verfeinert und durch die folgenden Schlüsselkomponenten beschrieben: \enquote{Parkhill examined the nature of utilities such as water, natural gas and electricity in the way they are provided to create an understanding of the characteristics that computing would require if it was truly a utility. When we consider electricity supply, for example, in the developed world, we tend to take it for granted that the actual electrical power will be available in our dwellings. To access it, we plug our devices into wall sockets and draw the power we need. Every so often we are billed by the electricity supply company, and we pay for what we have used}.\autocite[vgl.][]{parkhill_challenge_1966} Dieses Konzept leitete er auch auf eine technologische Ressource im Bereich des Computers ab.\autocite[vgl.][S.4]{hill_guide_2013} Der Gedanke der Serviceorientierung beschreibt eine klare Begrenzung einer Funktion, die zur Erfüllung eines bestimmten Ziels verwendet wird. Services werden meist durch die Konzepte der Objektorientierung und der Abstraktion in einer Organisation definiert. Aus dem Grundgedanken und den genannten Konzepten entwickelt sich die \ac{SOA}, die diese Prinzipien in ein technologiebasiertes Modell abbildet. Die Leitgedanken der \ac{SOA} spielen auch im \ac{Cloud-C} eine wichtige Rolle, denn, wie später noch näher definiert, ist der Servicegedanke ein elementarer Bestandteil der Cloud, der deutlich das Geschäftsmodell prägt. \enquote{Grid Computing} ist ein Konzept aus den 1990er-Jahren und fand seine Anwendung im Bereich der elektrischen Netze.\autocite[vgl.][]{weinhardt_cloud_2009} Ziel dieses Konzeptes war es, die Einfachheit und Zuverlässigkeit der Stromnetze zu gewährleisten über einen standardisierten Adapter Zugriff auf dieses zu erhalten ohne sich um die technische Realisierung kümmern zu müssen. Dabei stellten die Pioniere dieses Konzeptes folgende Eigenschaften\autocite[vgl.][]{foster_grid_1999} an das System:

\begin{itemize}
	\item Dezentrale Ressourcenkontrolle, d. h. ein Grid besteht aus geografisch verteilten Ressourcen, die administrativ unabhängig von Organisationen betreut werden.	
	\item Standardisierte, offene Protokolle und Schnittstellen, d. h. die Grid-Middleware
	ist nicht anwendungsspezifisch und kann zu verschiedenen Zwecken eingesetzt
	werden.
	\item Nichttriviale Eigenschaften des Dienstes, z. B. in Bezug auf Antwortzeitverhalten, Verfügbarkeit oder Durchsatz.
\end{itemize}
Diese Prinzipien haben eine Ähnlichkeit zu denen des \ac{Cloud-C}, jedoch sind die wirtschaftlichen Aspekte durch die Gedanken des \enquote{Grid Computing} beschrieben. Des weiteren werden die Aspekte des \enquote{Grid Computings} im Bereich des dezentralen Managements und der verteilten Ressourcen beim \ac{Cloud-C} nicht weiterverfolgt. Vielmehr bietet die Zentralisierung die ökonomischen Vorteile, die eine zentrale Rolle des Geschäftsmodells darstellen. \par
Da es mehrere Definitionen von \ac{Cloud-C} gibt, beschränkt sich diese Arbeit auf folgende: \enquote{Cloud computing is a model for enabling ubiquitous, convenient, on-demand network access to a shared pool of configurable computing resources (e.g., networks, servers, storage, applications, and services) that can be rapidly provisioned and released with minimal management effort or service provider interaction. This cloud model is composed of five essential characteristics, three service models, and four deployment models.}\autocite[][S.2]{mell_nist_2011} Das \ac{NIST} beschreibt in der Publikation \cite{mell_nist_2011} folgende essentielle Charakteristika\footnote{Jedoch werden diese Charakteristika in anderen wissenschaftlichen Ausarbeitungen um \enquote{multitenancy}, \enquote{service oriented} und \enquote{utility-based pricing} ergänzt.\autocite[vgl.][S.1]{institute_of_electrical_and_electronics_engineers_cloud_2011}}: 

\begin{itemize}
	\item on-demand self-service
	\item broad network access
	\item resource pooling
	\item rapid elasticity
	\item measured service
\end{itemize}
Des weiteren beschreibt die \ac{NIST} drei Servicemodelle, wie sich Unternehmen die Cloud zunutze machen können: \ac{SaaS}, \ac{PaaS} und \ac{IaaS}. Dabei wird \ac{SaaS} definiert als: \enquote{The capability  provided to the consumer is to use the provider’s applications running on a cloud infrastructure. [...] The consumer does not manage or control the underlying cloud infrastructure including network, servers, operating systems, storage, or even individual application capabilities, with the possible exception of limited user-specific application configuration settings.}\autocite[][S.2]{mell_nist_2011} Cloud-Infrastruktur ist eine Sammlung von Hard-/Software des Cloud-Anbieters, die die fünf essentiellen Charakteristika des \ac{Cloud-C} unterstützt bzw. erfüllt. Beispiele hierfür sind \textsc{Google Docs} und \textsc{Office 365}. \ac{PaaS} wird beschrieben durch: \enquote{The capability provided to the consumer is to deploy onto the cloud infrastructure consumer-created or acquired applications created using programming languages, libraries, services, and tools supported by the provider. The consumer does
not manage or control the underlying cloud infrastructure including network, servers, operating systems, or storage, but has control over the deployed applications and possibly configuration settings for the application-hosting environment.}\autocite[][S.2]{mell_nist_2011} Bei der später in der Konzeptionierung verwendeten Software, \textsc{OpenShift}, handelt es sich um eine \ac{PaaS}-Lösung. Weitere Beispiele sind 
\textsc{Google App Engine}, \textsc{Windows Azure} und \textsc{Heroku}.\autocite[vgl.][S.8]{kumar_reliability_2018} \ac{IaaS} wird durch folgende Definition abgebildet: \enquote{The capability provided to the consumer is to provision processing, storage, networks, and other fundamental computing resources where the consumer is able to deploy and run arbitrary software, which can include operating systems and applications. The consumer does not manage or control the underlying cloud infrastructure but has control over operating systems, storage, and deployed applications;and possibly limited control of select networking components (e.g. host firewalls).}\autocite[][S.3]{mell_nist_2011} Hierzu zählen die Produkte \textsc{Amazon EC2}, \textsc{OpenStack} und \textsc{VMware}. Nun sind die Bereitstellungsmodelle der Cloud noch von Bedeutung -- die \ac{NIST} sowie weitere, schon für diesen Abschnitt verwendete, Literatur definiert vier Modelle: \enquote{private, community, public and hybrid cloud} Die \enquote{private cloud} ist in exklusiver Nutzung eines Unternehmens, dass mehrere interne Konsumenten bedient. Es kann entscheiden, ob alle Management-/Betriebsoperationen intern oder extern von einem Anbieter durchgeführt werden. Die Cloud kann intern oder extern gehostet sein. Die \enquote{community cloud} ist eine \enquote{private cloud}, jedoch unterscheiden sich die beiden durch die Benutzergruppen. Bei der \enquote{community}-Variante ist es nicht auf Organisation sondern auf Gruppe mit gleichen Angelegenheiten beschränkt. Die \enquote{public cloud} ist offen für die Öffentlichkeit natürlich beschränkt durch die Regel des Cloud-Anbieter. Die hybride Variante wird folgendermaßen beschrieben: \enquote{The cloud infrastructure is a composition of two or more distinct cloud infrastructures (private, community, or public) that remain unique entities, but are bound together by standardized or proprietary technology that enables data and application portability (e.g., cloud bursting for load balancing between clouds).}\autocite[][S.3]{mell_nist_2011}
\subsection{Container}

\paragraph{Definition}

\paragraph{Grundgedanken und Architektur}

\paragraph{\textsc{Docker, Inc.} als Anbieter}



\subsection{\enquote{Deployment}} \label{defDeployment}

\section{Ist-Analyse des jetzigen \enquote{Deployment}-Prozesses}

\section{Konzeption eines container-basierten, automatisierten \enquote{Deployments}}

\section{Ergebnis der Forschungsfrage eins}
% business benefits cloud? --> Teilbereich