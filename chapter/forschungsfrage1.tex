\chapter[Forschungsfrage 1]{Wie können Container-Anwendungen den Prozess des automatisierten \enquote{Deployments} unterstützen?} \label{ff1}
Kapiteleinleitung...

\section{Grundlagen: Definieren der Begrifflichkeiten zur Forschungsfrage eins}
Dieses Kapitel soll grundlegende Begrifflichkeiten, die im weiteren Verlauf dieser Arbeit verwendet werden, definieren, um so eine einheitliche Terminologie der Begriffe zu entwickeln. Dadurch wird ein gemeinsames Verständnis erzeugt.

\subsection{Methodik der Anforderungsanalyse}
Die Anforderungsanalyse leitet sich aus der Disziplin \enquote{Requirements-Engineering} ab, die verschiedene Bedeutungsvarianten besitzt -- dabei \enquote{[...] steht [es] einmal für alle konkreten Aktivitäten am Beginn einer Systementwicklung, die auf eine Präzisierung der Problemstellung abzielen. Ebenso steht es aber auch für eine ganze Teildisziplin im Grenzbereich zwischen Systems-Engineering, Informatik und Anwendungswissenschaften.}\autocite[][S.19]{partsch_requirements-engineering_2010}

\subsection{Cloud Computing}

\subsection{Container}

\subsection{\enquote{Deployment}} \label{defDeployment}

\section{Ist-Analyse des jetzigen \enquote{Deployment}-Prozesses}

\section{Konzeption eines container-basierten, automatisierten \enquote{Deployments}}