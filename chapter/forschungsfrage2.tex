\chapter[Forschungsfrage 2]{Welche wirtschaftlichen Vorteile hat der Einsatz von Container auf den Prozess des automatisierten \enquote{Deployments}?} \label{ff2}
In diesem Kapitel ...

\section{Grundlagen: Definieren der Begrifflichkeiten zur Forschungsfrage zwei}
Dieses Teilkapitel soll grundlegende Begrifflichkeiten, die im weiteren Verlauf dieser Arbeit verwendet werden, definieren, um so eine einheitliche Terminologie der Begriffe zu entwickeln. Dadurch wird ein gemeinsames Verständnis erzeugt.

\subsection{Geschäftsprozessanalyse}
Der Begriff \enquote{Geschäftsprozess} beschreibt eine zusammenhängende Folge von Aufgaben beziehungsweise Tätigkeiten, die in einem Unternehmen abgeschlossen werden, um die Unternehmens-/Organisationsziele zu erreichen. Die Analyse untersucht schlussendlich die selben Sachverhalte, wie auch die klassischen Ansätze der Organisationslehre. \autocite[vgl.][S.5]{staud_geschaftsprozessanalyse_2006} Diese sind klassisch die Effizienzsteigerung und die Einsparung. Dabei werden die zu leistenden Tätigkeiten, Aufgaben und Arbeitsabläufe auf die genannten Ansätze optimiert. Im Vergleich zur klassischen Optimierung steht bei der Geschäftsprozessanalyse eine andere Perspektive im Fokus. Hier werden die \enquote{längere(n) zusammenhängende(n) Folgen von Tätigkeiten, die zur Erledigung einer größeren Aufgabe nötig sind}\autocite[][S.5]{staud_geschaftsprozessanalyse_2006}, betrachtet. Damit ist der gesamte Ablauf eines Prozesses als Ausgangspunkt der Analyse zu betrachten und nicht mehr nur einzelne Tätigkeiten und Stellen. \par 

Um das weitere Verständnis der Begrifflichkeiten zu fördern, werden folgende Begriffe definiert\autocite[vgl.][S.4-5]{staud_geschaftsprozessanalyse_2006}: Aufgaben und deren Eigenschaften, Aufgabenfolgen und Funktionen. Aufgaben sind Teilarbeitspakete einer Tätigkeit, die auf unterschiedlichen Ebenen betrachtet werden können. Die kleinste Einheit einer Aufgabe ist die Elementaraufgabe, die nicht weiter teilbar ist. Wichtig ist, dass Aufgaben teilbar und wieder zusammenfassbar sind, so wird eine unterschiedliche Aggregationsstufe erreicht. Das Problem der Aggregation ist, dass die Modelliererin, geprägt durch ihre Wahrnehmung, die Ebene der Betrachtung einer Aufgabe/Tätigkeit subjektiviert und so das Ergebnis stark beeinflusst wird -- so auch die Länge der Geschäftsprozesse. Die sequenzielle Folge von Aufgaben entsteht durch die Erstellung eines Vorgangs, der eine Abfolge von Tätigkeiten zur Realisierung von Aufgaben beschreibt. Schließlich wird ein Geschäftsprozess von \cite{staud_geschaftsprozessanalyse_2006} definiert als: \enquote{[...] besteht aus einer zusammenhängenden abgeschlossenen Folge von Tätigkeiten, die zur Erfüllung einer betrieblichen Aufgabe notwendig sind. Die Tätigkeiten werden von Aufgabenträgern in organisatorischen Einheiten unter Nutzung der benötigten Produktionsfaktoren geleistet. Unterstützt wird die Abwicklung der Geschäftsprozesse durch das Informations- und Kommunikationssystem IKS des Unternehmens.}\autocite[][S.9]{staud_geschaftsprozessanalyse_2006} Eine weitere Definition charakterisiert den Geschäftsprozess als \enquote{[...] eine zielgerichtete, zeitlich logische Abfolge von Aufgaben,die arbeitsteilig von mehreren Organisationen oder Organisationseinheiten unter Nutzung von Informations- und Kommunikationstechnologien ausgeführt werden können. Er dient der Erstellung von Leistungen entsprechend den vorgegebenen, aus der Unternehmensstrategie abgeleiteten Prozesszielen. Ein Geschäftsprozess kann formal auf unterschiedlichen Detaillierungsebenen und aus mehreren Sichten beschrieben werden.}\autocite[][S.41]{gadatsch_grundkurs_2010} Die zweite Definition wird in dieser Arbeit verwendet, denn sie stellt die Unternehmensstrategie als zentralen Messfaktor in den Mittelpunkt. Werden alle Geschäftsprozesse linear kombiniert, entsteht die Darstellung der Wertschöpfungskette eines Unternehmens. Deswegen gibt es nur systemrelevante Geschäftsprozesse in einem Unternehmen. Sie können noch Optimierungspotential enthalten, jedoch sind sie nie unnötig oder nicht brauchbar. Ein Geschäftsprozess wird an dem Kunden orientiert. Es wird trotzdem zwischen Kern- und unterstützenden Prozessen unterschieden: bei Kernprozessen handelt es sich um die Hauptleistung eines Unternehmens, wie die Produktion eines Autos bei einem Autohersteller. Die Unterteilung in Kern- und unterstützende Prozesse beschreibt dabei nicht die Wichtigkeit dieser; es ist also keine Einteilung in weniger wichtig und wichtig vorzunehmen.\autocite[vgl.][S.11]{staud_geschaftsprozessanalyse_2006} \par

Geschäftsprozesse haben verschiedene Eigenschaften, wie der Automatisierungsgrad, die Datenintegration und die Prozessintegration. Der Automatisierungsgrad beschreibt wie groß der Anteil der Aufgabenerfüllung ist, welcher dunkel, d. h. ohne menschliche Interaktion, bewältigt werden kann. Die Datenintegration ist ein wichtiger Bestandteil bei Optimierungsvorhaben, denn sie sollte bei 100 Prozent liegen, um Inkonsistenzen der Daten auszuschließen. Bei weniger als 100 Prozent entwickeln sich Parallelwelten im Unternehmen. Ist ein Geschäftsprozess über viele verschiedenen traditionelle Organisationsbereiche aufgespannt, so ist seine Prozessintegration hoch. Gibt es Organisationsbrüche, d. h. wird ein Prozess aktiv an einer beteiligten Abteilung vorbei geführt, muss die Notwendigkeit dieser Maßnahme bei der Optimierung überprüft werden. Zu den Komponenten der Geschäftsobjekte: Je nach Ziel der Untersuchung können beziehungsweise sind viele Komponenten beteiligt und damit identifiziert werden. Deswegen beschränkt die \ac{BWL} diese auf die formellen Strukturen einer Organisation und auf das Handeln der Beteiligten, das unmittelbar Einfluss auf den Geschäftsprozess hat.\autocite[vgl.][S.15]{staud_geschaftsprozessanalyse_2006} \par

Ziel der Geschäftsprozessanalyse ist es, eine IST-Analyse des Prozesses durchzuführen, um so eine Bestandsaufnahme vorhalten zu können, und eine Optimierung diesem, die die Beseitigung von Schwachstellen zur Folge hat. Diese werden bei der IST-Analyse entdeckt. Einschränkend ist die Methodik der Geschäftsprozessanalyse nicht genau definiert, da die Identifikation (Detaillierungsgrad) und Abgrenzung (Länge) der Prozesse subjektiv beeinflusst ist. Das Modell der \ac{EPK} ist die führende Methodik, um Geschäftsprozesse zu analysieren und zu beschreiben.\autocite[vgl.][S.59]{staud_geschaftsprozessanalyse_2006}
\subsection{\enquote{Business Case}}

\section{Wirtschaftliche Analyse des Geschäftsszenarios \enquote{Container Deployment}}

\section{Konzeption eines verbesserten Geschäftsszenarios}

\section{Ergebnis der Forschungsfrage zwei}
