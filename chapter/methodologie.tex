\chapter{Methodologie: Beschreibung des Vorgehens}
%Wie bin ich vorgegangen? Vorgehen auf Meta-Ebene beschreiben ...
%- Anforderungsanalyse
%- Aufbau Testumgebung 
%-- Deployment Test App

% ANMERKUNG FÜR RITA: IN VERGANGENHEIT ZU VERFASSEN!!!
\paragraph{Forschungsfrage eins}
Die Anforderungsanalyse, die in dieser Forschungsfrage benutzt wurde, verwendet das Vorgehen von \cite{hull_requirements_2011} und \cite{partsch_requirements-engineering_2010}. Vor allem in \cite{hull_requirements_2011} wurde die methodische Vorgehensweise stark fokussiert, um eine hohe Qualität und Aussagekraft der Analyse zu gewährleisten. Deswegen nutzte diese Arbeit die dort illustrierte Methodik, da, erstens, es ein de-facto-Standard darstellt\footnote{Die Monographie wurde 1200 mal laut \textsc{Google Scholar} -- \cite{google_llc_google_2020} -- zitiert.} und, zweitens, das Vorgehen passend für die Problemstellung dieser Arbeit erschien. Die \enquote{stakeholder} wurden anhand des vorhandenen \enquote{Deployment}-Prozesses abgeleitet, d. h. es war beim aktuellen Prozess Projektbeteiligte in Form von Rollen definiert. Diese wurden für jede Ausprägung des Prozess mit bestimmen \enquote{stakeholder} gefüllt. Dadurch war der relevante Personenkreis aus den aktuellen Prozessen auf den zu implementierenden übertragen. Das \enquote{statement of needs} war durch den Fachbereich \enquote{Entwicklung} konkretisiert. Des Weiteren sollte ein Fragebogen die Wünsche der beteiligten Fachbereiche (\enquote{stakeholder}) aufnehmen, um daraus die präzisen Anforderungen abzuleiten. Dabei ward absichtlich eine offene Fragestellung an den Entwicklungs-Fachbereich (namentlich: \enquote{\textsc{Squad1}}) gesendet, da es keine Beeinflussung durch die Fragestellung geben sollte. Ziel war es, die wirklichen Wünsche des Fachbereichs herauszufinden. Aus den Wünschen wurden nach dem Muster von \cite[vgl.][S.28]{hull_requirements_2011} (siehe dazu auch Kapitel \vref{kap:methodikAnfAnalyse}) spezifische funktionale und nicht-funktionale Anforderungen formuliert. Als Formulierungshilfe, um die Verständlichkeit und die Messbarkeit zu waren, dienten die Regeln nach \cite{rupp_formulierungsregel_2020}. 
\par
Die Modellierung einer \enquote{\textsc{OpenShift}}-Labor-Umgebung diente dem Zweck der Anwendungsevaluierung, so wurde die Funktionsweise dieses System unter eingeschränkten Bedingungen getestet. Einschränkend wirkten die Limitierung auf eine \enquote{worker node}, die beschränkte Leistung der virtuellen Maschine, die Version von \textsc{OpenShift} und die Entkopplung von der internen \ac{AWL}. Hier wurde die \enquote{open source}-Variante genutzt, da das Unternehmen die \enquote{Entreprise}-Edition zum Zeitpunkt der Evaluation nicht-lauffähig präsentierte. Dieser Faktor verhinderte den Abschlusstest des modellierten Prozesses. Jedoch konnte durch die Labor-Umgebung die allgemeine Funktionsweise des Clusters getestet werden. Ziel dieser Umgebung war es, Tests zu ermöglichen und das Zusammenspiel der einzelnen Komponenten zu überprüfen. Aus diesem Grund wurde die Methodik der Labor-Umgebung gewählt, da sie einige Vorteile für diese Arbeit bot: Es konnte die Funktionsweise der Anwendungen getestet werden, die Konfigurationsdateien für das Deployment wurden unter realistischen Bedingungen erforscht und die Interaktion mit anderer Hilfssoftware ist evaluiert worden. Die Erkenntnisse wurden, soweit möglich, auf die \ac{AWL} übertragen. Problematisch war es, dass die kompletten Abhängigkeiten der einzelnen \enquote{\ac{AWL}}-Bestandteile in der Labor-Umgebung nicht vollständig modelliert wurden.
\par
% Prozessmodellierung
Die Prozessmodellierung des Container-\enquote{Deployments} nutze die Erkenntnisse aus der Anforderungsanalyse und der Labor-Umgebung, um einen möglichst stark an die Anforderungen angepassten Prozess zu erzeugen. Hierbei wurde das Sequenzdiagramm\footnote{Eine Art des Interaktionsdiagramms mit Fokus auf den Nachrichtenaustausch; spezifiziert in \cite[][S.595-599]{object_management_group_omg_unified_2017}.} der \ac{UML} in einer adaptierten Form zur Visualisierung des Prozesse genutzt. Folgende Anpassungen wurden vorgenommen: die \enquote{Lifeline} einer Instanz der Klasse stellte einen Prozessbeteiligten jeglicher Art dar. Sonst blieben die Semantik alle anderen Bestandteile unverändert. Das Sequenzdiagramm wurde gewählt, da es für die Beschreibung des Nachrichtenaustausches von verschieden Objekten benutzt wird. Dies war genau das darzustellende Ziel der Prozessmodellierung. Durch dieses Vorgehen wurde die Prozessmodellierung anwendungsunabhängig erstellt, um so dem Gedanken der Container\footnote{Hiermit ist das Desinteresse der \enquote{Deployment}-Abteilung an dem Inhalt des Containers gemeint, d. h. es sollte dem  \enquote{Deploy-Prozess} egal sein, was der Container enthält.} gerecht zu werden. 
\par
% vielleicht Pseudo-Code für das Ergebnis nutzen, d. h. das Programm zur Generierung einer config-Datei
Mustererkennung 

\paragraph{Forschungsfrage zwei}
