\chapter[Forschungsfrage 3]{Welche besonderen sicherheitstechnischen Aspekte muss ein solcher Prozess im Bereich der Versicherung erfüllen?} \label{ff3}
Diese Kapitel ... \par
Informations- und Kommunikationssysteme sind in der heutigen Gesellschaft von elementarer Bedeutung -- sie spielen eine immer größer werdende Rolle. Der Innovationsgrad in der Informationstechnik ist konstant hoch und deswegen sind folgende Bereiche ständiger Weiterentwicklung unterlegen: steigende Vernetzung der Bevölkerung, IT-Verbreitung und Durchdringung, verschwinden der Netzgrenzen, kürze Angriffszyklen auf wichtige Infrastruktur, höhere Interaktivität von Anwendungen und die Verantwortung der Benutzer eines IT-Systems.\autocite[vgl.][S.2f.]{bundesamt_fur_sicherheit_in_der_informationstechnik_bsi_it-grundschutz-kompendium_2020}

\section{Sicherheitstechnische Anforderungen an den Betrieb einer Anwendung}
Informationen sind elementarer Bestandteil der heutigen Welt -- diese sind von sehr hohem Wert für Unternehmen und Behörden. Die meisten Geschäftsprozesse, die im heutigen Prozessablauf einer Organisation verankert sind, funktionieren nicht ohne IT-Unterstützung. Somit ist die Informationstechnologie elementarer Bestandteil jedes Unternehmens. Deswegen ist ein zuverlässiges System mit entsprechender Soft- und Hardware unerlässlich. Es muss darauf geachtet werden, dass die Informationen, die auf diesen System verteilt sind, ausreichend gut geschützt sind, damit es nicht zu einer Bedrohungslage kommt. Unzureichend geschützte Systeme stellen ein sehr hohes Risiko dar. \enquote{Dabei ist ein vernünftiger Informationsschutz ebenso wie eine Grundsicherung der IT schon mit verhältnismäßig geringen Mitteln zu erreichen. Die verarbeiteten Daten und Informationen müssen adäquat geschützt, Sicherheitsmaßnahmen sorgfältig geplant, umgesetzt und kontrolliert werden. Hierbei ist es aber wichtig, sich nicht nur auf die Sicherheit von IT-Systemen zu konzentrieren, da Informationssicherheit ganzheitlich betrachtet werden muss. Sie	hängt auch stark von infrastrukturellen, organisatorischen und personellen Rahmenbedingungen ab. }\autocite[][S.1]{bundesamt_fur_sicherheit_in_der_informationstechnik_bsi_it-grundschutz-kompendium_2020} Die Mängel in der IT-Sicherheit führen meist zu folgenden drei Kategorien von Problemen\autocite[vgl.][S.1ff.]{bundesamt_fur_sicherheit_in_der_informationstechnik_bsi_it-grundschutz-kompendium_2020}: 

\begin{itemize}
	\item Verlust der Verfügbarkeit
	\item Verlust der Vertraulichkeit
	\item Verlust der Integrität
\end{itemize}

Der Verlust der Verfügbarkeit eines IT-Systems fällt \ac{i.d.R.} sofort auf, da meist Aufgaben ohne diese Informationen nicht weitergeführt werden können. Meist fällt dies in dem Verlust der Funktionen eines Systems auf. Die Vertraulichkeit von personenbezogenen Daten ist ein bestehendes Grundrecht jedes Bürgers beziehungsweise jedes Kunden. Dies ist in verschiedenen Gesetzen wie auch Verordnung geregelt. Diese Daten müssen geschützt werden, da jedes Konkurrenzunternehmen Interesse an den Daten des Unternehmens hat. \enquote{Gefälschte oder verfälschte Daten können beispielsweise zu Fehlbuchungen, falschen Lieferungen oder fehlerhaften Produkten führen. Auch der Verlust der Authentizität (Echtheit und Überprüfbarkeit) hat, als ein Teilbereich der Integrität, eine hohe Bedeutung: Daten werden beispielsweise einer falschen Person zugeordnet. So können Zahlungsanweisungen oder Bestellungen zulasten einer dritten Person verarbeitet werden, ungesicherte digitale Willenserklärungen können falschen Personen zugerechnet werden, die digitale Identität wird	gefälscht.}\autocite[][S.1]{bundesamt_fur_sicherheit_in_der_informationstechnik_bsi_it-grundschutz-kompendium_2020}

\subsection{IT-Sicherheit: Grundnorm ISO 27001}

\subsection{IT-Grundschutz-Katalog}

\subsection{\ac{VAIT}}

\section{Beschaffung von \enquote{open source}-Software}
In der \ac{SVI} gibt es, wie in den meisten anderen Unternehmen, eine prozessorientierte Vorgehensweise, um Software zu beschaffen. Die Beschaffung von Software orientiert sich an \ac{ITIL} Version 4 -- formal ist die Beschaffung von Software mit Hilfe eines \enquote{service requests}\footnote{a request from a user or a user's authorized representative that initiates a service action which has been agreed as a normal part of service delivery. Quelle: \cite[][S.195]{axelos_limited_itil_2019}} zu beantragen. Für die Verteilung der Anwendung müssen danach mehrere \enquote{changes} eingereicht werden. Im weiteren Verlauf wird die \enquote{open source}\footnote{ausführliche Definition \vref{tab:definitionen}}-Variante beleuchtet, da es sich bei den verwendeten Containern, die von \textsc{Docker Inc.} angeboten werden, um diese Variante handelt. Definitionsgemäß muss \enquote{open source}-Software laut \cite{opensource.org_open_2020} folgende Kriterien erfüllen: \enquote{free redistribution, source code, derived works, integrity of the author's source Code, no discrimination against persons or groups, no discrimination against fields of endeavor, distribution of license, license must not be specific to a product, license must not restrict other software, license must be technology-neutral}. \par
Es gibt in der \ac{SVI} drei Prozesse, die sich in zwei Aspekten unterscheiden: die Kosten und die Anforderungen, die an einen Prozess gestellt werden. Folgende Anfragen gibt es: die Beschaffungsanfrage, die \enquote{freeware}-Beschaffung und die juristische Prüfung von Vertragsdokumenten oder Sachverhalten. Die Beschaffungsanfrage wird bei kostenpflichtiger Software beantragt. Da es in diesem Kapitel um die kostenlose Software geht, wird auf die weitere Ausführung dieser Anfrage verzichtet. Der Prozess \enquote{freeware}-Beschaffung wird laut den Juristen der Abteilung \ac{IU11} kaum\footnote{$ n \leq 5, n \in \mathbb{N}_{0} $, gemessen p. a.} verwendet, denn die Fachbereiche\footnote{aus Sicht von \ac{IU11}} (die IT-Abteilungen) arbeiten zum jetzigen Zeitpunkt an dem Prozess vorbei -- sie übergehen wissentlich diesen. Folgende Probleme haben sich bei der Befragung der Fachbereiche herausgestellt: die Anforderungen, die dieser Prozess an sie stellt, sind \enquote{nicht verhältnismäßig} gegen über dem Nutzen; die Fachbereiche wissen nicht, dass es einen solchen Prozess gibt oder ignorieren diesen. Die Anforderungen/Kriterien, die die Abteilung \ac{IU11} festgelegt hat, sind folgende: es muss eine Produktverantwortliche definiert werden, es muss eine Architekturfreigabe von den zuständigen \enquote{Entreprise}-Architekten beantragt werden und es muss der genaue, angedachte Verwendungszweck der einzukaufenden \enquote{freeware/open source}-Software definiert werden. Diese Hürden, aus Sicht der IT-Abteilung, erfüllen nicht die Kosten-Nutzen-Konformität. Aus rechtlicher Sicht ist das ein sehr hoch zu bewertendes Risiko, da es zu unmittelbaren juristischen Konsequenzen führen kann. Deswegen nutzt die IT-Abteilung meist den rechtlichen Prozess (juristische Prüfung von Vertragsdokumenten oder Sachverhalten) da dieser nicht die oben genannten Hürden enthält. Bei diesem wird der Verwendungszweck der Software erfragt und die Lizenz dieser durch \ac{IU11} geprüft. Jedoch ist davon auszugehen, dass eine offizielle Beschaffungsanfrage bei \enquote{open source}-Software in wenigen\footnote{$ n \leq 10, n \in \mathbb{N}_{0} $, gemessen p. a.} Fällen gestellt wird. Begründet durch die Administrator-Berechtigung, die es Benutzern erlaubt ohne Restriktionen alles auf ihrem Computer zu installieren, kann keine numerische Aussage über die Dunkelziffer getroffen werden. Es bleibt nur die Hypothese der Juristen der Abteilung \ac{IU11}, die weder falsifizierbar noch validierbar ist. \par
Ist die Software in der \ac{AWL} implementiert, gibt es noch eine Anwendung, \textsc{Nexus Lifecycle} von \textsc{sonatype}, die auf eventuelle Schwachstellen dieser benutzten Software prüft. \textsc{Nexus Lifecycle} ist eine Hilfsanwendung, die u. a. auch von \textsc{Creditreform} verwendet wird. Das Ziel dieses Produktes ist es, die gesamte Software-\enquote{Supply Chain} kontinuierlich zu bereinigen und sicher zu halten.\autocite[vgl.][]{sonatype_inc._nexus_2020} Aus dem Prüfbericht werden dann entsprechende Maßnahmen abgeleitet. Die erste ist die Software, in der die Schwachstelle gefunden wurde, als unsicher zu markieren und danach zu sperren. Nun müssen die Entwicklungsabteilung versuchen die Schwachstellen zu beseitigen. Problematisch ist es, wenn diese ignoriert werden. In letzter Konsequenz wird der Betrieb und die Verteilung der Anwendung gestoppt. Dies führt zu massiven Problem in der Produktion und somit verringert sich die vertragliche, mit dem Kunden vereinbarte, Verfügbarkeit der Systeme.

\section{Konzept zur Implementierung der Sicherheitsanforderungen}

\section{Ergebnis der Forschungsfrage drei}


