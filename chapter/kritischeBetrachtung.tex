\chapter{Epilog} \label{kritischeBetrachtung}
Die Bachelorthesis schließt mit dem Epilog ab. Hier werden die Teilergebnisse der Forschungsfragen eins bis drei zusammengefasst und eine abschließende Erkenntnis aus diesen abgeleitet. 
\paragraph{Zusammenfassung der Erkenntnisse}
Hier werden die einzelnen Ergebnisse der Forschungsfragen subsumiert und eine abschließende Erkenntnis abgeleitet. Diese unterliegen den Beschränkungen des in der Methodologie (siehe Kapitel \vref{kap:methodology}) festgelegten Vorgehens. Mit der Festlegung des Vorgehens und der Ziele bzw. der Problemstellung (siehe Kapitel \vref{kap:einleitung}) sind die Erkenntnisse in die vorgegebene Richtung gelenkt worden, so konnte das Forschungsobjekt identifiziert und analysiert werden.
\par
Forschungsfrage eins beschäftigte sich mit der Frage: \enquote{Wie können Container-Anwendungen den Prozess des automatisierten \enquote{Deployments} unterstützen?} Diese

\paragraph{Fazit}
Überprüfung der Zieleinhaltung
\paragraph{Ausblick}