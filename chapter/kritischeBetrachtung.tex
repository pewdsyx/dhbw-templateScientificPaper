\chapter{Epilog} \label{kritischeBetrachtung}
Die Bachelorthesis schließt mit dem Epilog ab. Hier werden die Teilergebnisse der Forschungsfragen eins bis drei zusammengefasst und eine abschließende Erkenntnis aus diesen abgeleitet. 
\paragraph{Zusammenfassung der Erkenntnisse}
Hier werden die einzelnen Ergebnisse der Forschungsfragen subsumiert und eine abschließende Erkenntnis abgeleitet. Diese unterliegen den Beschränkungen des in der Methodologie (siehe Kapitel \vref{kap:methodology}) festgelegten Vorgehens. Mit der Festlegung des Vorgehens und der Ziele bzw. der Problemstellung (siehe Kapitel \vref{kap:einleitung}) sind die Erkenntnisse in die vorgegebene Richtung gelenkt worden, so konnte das Forschungsobjekt identifiziert und analysiert werden.
\par
Die Forschungsfrage eins beschäftigte sich mit der Frage: \enquote{Wie können Container-Anwendungen den Prozess des automatisierten \enquote{Deployments} unterstützen?} Diese Frage identifizierte verschiedene Erkenntnisse: der Prozess wird generisch bzw. generischer, der Aufwand ist langfristig geringer, die Kontrolle des Erfolgs ist verbessert, jedoch ist der Generierungsalgorithmus \vref{algo:configGenerierung} nicht optimiert. Die Optimierung des Algorithmus im Sinne der Komplexität und der Laufzeit des Algorithmus ist notwendig, um diesen produktiv einsetzen zu können. Abschließend ist zu erwähnen, dass der komplette \enquote{Deployment}-Prozess einer ständigen Weiterentwicklung bedarf.
\par
Die Forschungsfrage zwei untersuchte die Frage: \enquote{Welche wirtschaftlichen Vorteile hat der Einsatz von Containern auf den Prozess des automatisierten \enquote{Deployments}?} Das Ergebnis der Frage erzeugt folgende Erkenntnisse: die Erkenntnis, dass die \enquote{Business case}-Methodik nicht vollständig im Unternehmen umgesetzt wurde, führte dazu, dass ein Soll-Ist-Vergleich die Unterschiede der Theorie und der Praxis hervorheben konnte; die wirtschaftlichen Auswirkungen der Container-Anwendungen, dabei wird auf die schnelle \ac{TTM}-Rate als großer Vorteil beschrieben.
\par
Die Forschungsfrage drei beschäftigt sich mit der Frage: \enquote{Welche besonderen sicherheitstechnischen Aspekte muss ein solcher Prozess im Bereich der Versicherung erfüllen?} Dabei sind stellten sich folgende Erkenntnisse nach der Analyse ein: Da die Testumgebung eine Labor-Umgebung war und somit nicht alle Funktionen und Sicherheitsbestimmungen wie in der Produktion implementiert waren, wird eine abschließende Sicherheitsprüfung bei beim Produktivgang empfohlen und die Bewertung durch die Gremien der \ac{SVI} und \ac{SV} bezüglich der selbsterstellten Risikobewertung sollten bei Übernahme der Anwendung in die \ac{AWL} neu durchgeführt werden.

\paragraph{Fazit}
Überprüfung der Zieleinhaltung
\paragraph{Ausblick}