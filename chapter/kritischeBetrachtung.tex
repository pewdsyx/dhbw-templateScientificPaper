\chapter{Epilog} \label{kritischeBetrachtung}
Die Bachelorthesis schließt mit dem Epilog ab. Hier werden die Teilergebnisse der Forschungsfragen eins bis drei zusammengefasst und eine abschließende Erkenntnis aus diesen abgeleitet. 
\paragraph{Zusammenfassung der Erkenntnisse}
Hier werden die einzelnen Ergebnisse der Forschungsfragen subsumiert und eine abschließende Erkenntnis abgeleitet. Diese unterliegen den Beschränkungen des in der Methodologie (siehe Kapitel \vref{kap:methodology}) festgelegten Vorgehens. Mit der Festlegung des Vorgehens und der Ziele bzw. der Problemstellung (siehe Kapitel \vref{kap:einleitung}) sind die Erkenntnisse in die vorgegebene Richtung gelenkt worden, so konnte das Forschungsobjekt identifiziert und analysiert werden.
\par
Die Forschungsfrage eins beschäftigte sich mit der Frage: \enquote{Wie können Container-Anwendungen den Prozess des automatisierten \enquote{Deployments} unterstützen?} Diese Frage identifizierte verschiedene Erkenntnisse: der Prozess wird generisch bzw. generischer, der Aufwand ist langfristig geringer, die Kontrolle des Erfolgs ist verbessert, jedoch ist der Generierungsalgorithmus \vref{algo:configGenerierung} nicht optimiert. Die Optimierung des Algorithmus im Sinne der Komplexität und der Laufzeit des Algorithmus ist notwendig, um diesen produktiv einsetzen zu können. Abschließend ist zu erwähnen, dass der komplette \enquote{Deployment}-Prozess einer ständigen Weiterentwicklung bedarf.
\par
Die Forschungsfrage zwei untersuchte die Frage: \enquote{Welche wirtschaftlichen Vorteile hat der Einsatz von Containern auf den Prozess des automatisierten \enquote{Deployments}?} Das Ergebnis der Frage erzeugt folgende Erkenntnisse: die Erkenntnis, dass die \enquote{Business Case}-Methodik nicht vollständig im Unternehmen umgesetzt wurde, führte dazu, dass ein Soll-Ist-Vergleich die Unterschiede der Theorie und der Praxis hervorheben konnte; die wirtschaftlichen Auswirkungen der Container-Anwendungen, dabei wird auf die schnelle \ac{TTM}-Rate als großer Vorteil beschrieben.
\par
Die Forschungsfrage drei beschäftigt sich mit der Frage: \enquote{Welche besonderen sicherheitstechnischen Aspekte muss ein solcher Prozess im Bereich der Versicherung erfüllen?} Dabei sind stellten sich folgende Erkenntnisse nach der Analyse ein: Da die Testumgebung eine Labor-Umgebung war und somit nicht alle Funktionen und Sicherheitsbestimmungen wie in der Produktion implementiert waren, wird eine abschließende Sicherheitsprüfung bei beim Produktivgang empfohlen und die Bewertung durch die Gremien der \ac{SVI} und \ac{SV} bezüglich der selbsterstellten Risikobewertung sollten bei Übernahme der Anwendung in die \ac{AWL} neu durchgeführt werden.

\paragraph{Fazit}
In diesem Teilabschnitt sollen die formulierten Ziele (vgl. Kapitel \vref{kap:einleitung:Ziele}) aus der Einleitung evaluiert werden. Nachfolgend sind die Ziele nochmals aufgelistet:

\begin{enumerate}
	\item Entwicklung eines Verteilungsprozesses für einfache Container-Anwendungen bis zum 27.~April 2020. Einfach bedeutet hier, dass die Container-Struktur aus einer \enquote{Base Image}\footnote{siehe dazu Kapitel \vref{kap:container}}-Schicht und einer Logik-Schicht (Eigenentwicklung) besteht.
	\item Der Prozess muss zu 98\,\% ohne Einwirkung von Menschen während des Verteilungsvorgangs funktionieren, d.\,h., er ist (annähernd) voll automatisiert. Dies muss bis zum Ende des Bearbeitungszeitraums der Arbeit (8.~Mai 2020) umgesetzt werden.
	\item Die Generierung einer Konfigurationsdatei soll in 9 von 10 Verteilungen automatisch mit einem Skript durchgeführt werden. Umzusetzen ist dies bis zum 08.~Mai 2020.
	\item Die Vorteile einer Container-Anwendung für die Verteilung dieser sollen erforscht werden. Akzeptiert ist dieses Ziel, sobald eine Auflistung und eine kritische Betrachtung der Ergebnisse beschrieben wurde. Umzusetzen ist dies bis zum Ende des Bearbeitungszeitraums. (Dieses Ziel ist nicht komplett \textit{SMART}-konform, da zumindest die Messbarkeit ohne genannte Messgröße nicht nachvollziehbar ist.)
	\item Die wirtschaftliche Betrachtung muss sich am Stand von Wissenschaft und Technik orientieren. Dazu werden gängige Regeln von \cite{herman_is_2009} und \cite{brugger_it_2009} benutzt. Diese Betrachtung muss bis zum Ende des Bearbeitungszeitraums durchgeführt werden. Die Messbarkeit wird durch die Einhaltung der oben genannten Regeln beschrieben.
	\item Die sicherheits- und rechtlich-relevanten Aspekte dieses Projektes sollen anhand der für die Finanzdienstleitungs- und Versicherungsbranche geltenden Vorschriften beleuchtet werden. Der Umfang dieser Beleuchtung beschränkt sich auf die wichtigsten Bestandteile, d.\,h., es müssen nur die wichtigsten Regeln beschrieben werden. Dies ist bis zum Ende des Bearbeitungszeitraumes umzusetzen. Die Messbarkeit wird durch die sicherheits- und rechtlich-relevanten Vorschriften bestimmt.
\end{enumerate}

Das erste Ziel wurde durch die Forschungsfrage eins vollständig erfüllt. Es wurde ein neuer generischer \enquote{Deployment}-Prozess entwickelt, der einfache (Bedeutung: siehe Ziel 1) Container-Anwendungen verteilen kann. Der Prozess wurde termingerecht implementiert und befindet sich in der Testphase. Es folgen noch einige Sicherheitstests. Das zweite Ziel kann quantitativ nicht bewertet werden, da die eine Produktivumgebung bis zum Ende der First nicht mit dem neuen Prozess bespielt wurde. Qualitativ ist anzumerken, dass der Prozess so implementiert wurde, dass er mit dem Standard-\enquote{Deployment} ohne Fehler umgehen kann, d.\,h. es ist möglich ohne menschliche Einwirkung eine Verteilung von Container-Anwendungen durchzuführen. Dies bleibt noch durch quantitativ durch \enquote{Release}-Gänge zu messen. Das dritte Ziel wurde vollständig erfüllt, da der Algorithmus \vref{algo:configGenerierung} eine Konfigurationsdatei vollautomatisch erstellt. Das vierte Ziel ist erfüllt, da die technischen wie auch wirtschaftlichen Vorteile von Container-Anwendungen beleuchtet wurden. Die Erkenntnisse dazu sind ein erster Schritt im Unternehmen die Vorteile dieser vollständig zu implementieren. Das fünfte Ziel ist teilweise erfüllt, da sich die Analyse an der genannten Literatur orientiert, jedoch können die Vorgaben in der tatsächlichen Betrachtung nur im Form eine Soll-Ist-Analyse implementiert werden. Die Entscheidung über die Wirtschaftlichkeit der Container-Anwendung konnte nicht gestellt werden für den Kontext der \ac{SV} bzw. \ac{SVI}, da die Container-Lösung durch den Kauf einer anderen Software zwingend erforderlich ist. Das sechste Ziel wurde erfüllt. Es sind allgemeine sicherheitstechnische Aspekte beleuchtet worden, die mittleren Schutz bieten.  
\paragraph{Ausblick}
Wichtig ist, dass der neu entwickelte Prozess der Forschungsfrage eins noch weiterentwickelt und ständig verbessert wird, so müssen der Generierungsalgorithmus, der Prozess als solches und die Verantwortlichkeiten noch weiter optimiert bzw. definiert werden. Auch ist die wirtschaftliche Betrachtung formal nicht korrekt. Sie genügt sie den Anforderungen der \ac{SV} bzw. der \ac{SVI}, jedoch nicht den genannten Literaturquellen. Die dargestellten Vorteile der Container-Anwendungen werden erst im Produktivsystem der \ac{SVI} erkennbar sein, da dieses System zu Hauptbetriebszeiten eine derzeitige Auslastung von 8000-10.000\footnote{Quelle: interne Unternehmensauswertung} Benutzerinnen hat. Die Erleichterung des \enquote{Deployments} wird beim ersten \enquote{Release} messbar sein. So können nach jetzigem Stand nur die Labor-Umgebung bewertet werden. Die Sicherheitsvorkehrung sind nach dem Produtivgang ständig zu prüfen und anzupassen.

