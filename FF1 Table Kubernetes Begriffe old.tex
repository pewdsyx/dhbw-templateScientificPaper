\begin{table*}[H]
	\centering
	\ra{1.3} more space beetween rules
	
	\begin{tabular}{@{}lp{12.0cm}@{}}\toprule[1.5pt]
		
		\textbf{Begriff} & \textbf{Definition}  \\ \midrule
		% below rules with content
		
		\enquote{Pod} & \enquote{A Pod is a group of one or more tightly coupled Containers sharing a set of Linux namespaces and cgroups.} Innerhalb des \enquote{Pod} wird der Netzwerk-/\enquote{Mount}-Namensraum geteilt, dies ermöglicht die Kommunikationen innerhalb eines \enquote{Pod}. Mehrere dieser kommunizieren über \textit{localhost} \\
		
		\enquote{Services} &  \enquote{A Service is a Kubernetes object that maps one or more incoming ports to targetPorts at a selected set of target of Pods. These represent a microservice or an application running on the cluster.} \\
		
		\enquote{Master nodes} & \enquote{The master nodes are the nodes hosting core elements of the control plane like (not an exhaustive list) the kubeapi-server, kube-scheduler, kube-controller-manager, and in many	instances the etcd database.} Diese übernimmt die Management-Aufgaben im Cluster. \\
		
		\enquote{Worker nodes} & \enquote{The worker nodes (formerly known as minions) host elements like the kubelet, kube-proxy, and the container runtime.} Darin sind die \enquote{Pods} enthalten, welche die Container-Anwendungen beinhalten. \\
		
		\bottomrule[1.5pt]
	\end{tabular}
	
	\caption{Definition der \enquote{\ac{K8s}}-Begrifflichkeiten}
	{\footnotesize{Quelle: \cite[][S.\,10-14]{caban_architecting_2019}}}
	\label{tab:definitionenK8s}
	
\end{table*}

\begin{longtable}{@{}lp{12.0cm}@{}}
	\toprule[1.5pt]
	\textbf{Begriff} & \textbf{Definition} \\* \midrule
	\endfirsthead
	%
	\multicolumn{2}{c}%
	{{\bfseries Tabelle \thetable\ von vorheriger Seite fortgeführt.}} \\
	\toprule
	\textbf{Begriff} & \textbf{Definition} \\* \midrule
	\endhead
	%
	\bottomrule
	\endfoot
	%
	\endlastfoot
	%
	\enquote{Pod}              & \enquote{A Pod is a group of one or more tightly coupled Containers sharing a set of Linux namespaces and cgroups.} Innerhalb des \enquote{Pod} wird der Netzwerk-/\enquote{Mount}-Namensraum geteilt, dies ermöglicht die Kommunikationen innerhalb eines \enquote{Pod}. Mehrere dieser kommunizieren über \textit{localhost} \\
	\enquote{Services}         & \enquote{A Service is a Kubernetes object that maps one or more incoming ports to targetPorts at a selected set of target of Pods. These represent a microservice or an application running on the cluster.} \\
	\enquote{Worker nodes}     & \enquote{The worker nodes (formerly known as minions) host elements like the kubelet, kube-proxy, and the container runtime.} Darin sind die \enquote{Pods} enthalten, welche die Container-Anwendungen beinhalten. \\
	\enquote{Master nodes}     & \enquote{The master nodes are the nodes hosting core elements of the control plane like (not an exhaustive list) the kubeapi-server, kube-scheduler, kube-controller-manager, and in many	instances the etcd database.} Diese übernimmt die Management-Aufgaben im Cluster. \\* \bottomrule
	
	\caption{Definition der \enquote{\ac{K8s}}-Begrifflichkeiten}\label{tab:definitionenK8s}\\
\end{longtable}




\begin{table*}[h!]
	\centering
	\ra{1.3} %more space beetween rules
	
	\begin{tabular}{@{}lp{8.0cm}@{}}\toprule[1.5pt]
		
		\textbf{Komponente} & \textbf{Bausteine des IT-Grundschutzes}  \\ \midrule
		
		% below rules with content
		\enquote{bare metal}-Server & SYS.1.1 Server \\
		virtualisierte Server & SYS.1.5 Virtualisierung \\
		\textsc{Linux}-basierte Betriebssystem & SYS.1.3 Server unter Linux und Unix\\
		\textsc{OpenShift}-Anwendung & APP.1 Client-Anwendungen bzw. APP.3 Netz-basierte Dienste \\
		Netzwerk bzw. demilitarisierte Zone & NET.1 Netze und NET.3.2 \enquote{Firewall}\\
		Rechenzentrum & INF.2 Rechenzentrum sowie Serverräume\\
		
		\bottomrule[1.5pt]
	\end{tabular}
	
	\caption{Zuordnung der Komponenten zu den Bausteinen}
	\label{tab:zuordnungKompBau}
	
\end{table*}




\begin{table*}[h!]
	\centering
	\ra{1.3} %more space beetween rules
	
	\begin{tabular}{@{}p{5.5cm}p{8.0cm}@{}}\toprule[1.5pt]
		
		\textbf{Methode} & \textbf{Beschreibung} \\ \midrule
		% below rules with content

		
		push(Ü-Datei) & Veröffentlicht die Übergabedatei in der Versionsverwaltung. \\
		holeDatei(Datei) & Lädt die angegebene Datei von der Versionsverwaltung auf einen Server, der die Methode aufgerufen hat. \\
		starteVerteilung(Konfig-Vorlage, Ü-Datei) & Damit wird die Erstellung der angereicherten Konfigurationsdatei gestartet. \\
		sende(Konfig-Datei, Server) & Diese Methode versendet die Konfigurationsdatei an den entsprechend Server. \\
		validieren(Konfig-Datei) & Überprüft die Konfigurationsdatei auf Syntax-Fehler. \\
		hochladen(Konfig-Datei) & Lädt die Datei in \textsc{OpenShift}-Cluster hoch.\\
		stelleBereit(für wen?, Datei) & Diese Methode schaltet die angegebene Datei für alle berechtigten Benutzerinnen auf dem Cluster frei.\\
		setzeSichtbarkeit(Sichtbarkeit) & Veröffentlicht die Komponenten, die durch die Konfigurationsdatei beschrieben wurden. Damit ist der Verteilungsvorgang abgeschlossen.\\
		abbruch(Fehlermeldung) & Bricht den Verteilungsprozess ab und gibt eine Fehlermeldung zurück.\\
		
		\bottomrule[1.5pt]
	\end{tabular}
	
	\caption{Überblick über die verwendeten Methoden in Abbildung \vref{abb:prozessDeploymentCamunda}}
	\label{tab:methodenFunktion}
	
\end{table*}

\begin{longtable}{@{}p{5.5cm}p{8.0cm}@{}}
	\toprule[1.5pt]
	\textbf{Methode} & \textbf{Beschreibung} \\* \midrule
	\endfirsthead
	%
	\multicolumn{2}{c}%
	{{\bfseries Tabelle \thetable\ von vorheriger Seite fortgeführt.}} \\
	\toprule
	\textbf{Methode} & \textbf{Beschreibung} \\* \midrule
	\endhead
	%
	\bottomrule
	\endfoot
	%
	\endlastfoot
	%
	push(Ü-Datei) & Veröffentlicht die Übergabedatei in der Versionsverwaltung. \\
		holeDatei(Datei) & Lädt die angegebene Datei von der Versionsverwaltung auf einen Server, der die Methode aufgerufen hat. \\
		starteVerteilung(Konfig-Vorlage, Ü-Datei) & Damit wird die Erstellung der angereicherten Konfigurationsdatei gestartet. \\
		sende(Konfig-Datei, Server) & Diese Methode versendet die Konfigurationsdatei an den entsprechend Server. \\
		validieren(Konfig-Datei) & Überprüft die Konfigurationsdatei auf Syntax-Fehler. \\
		hochladen(Konfig-Datei) & Lädt die Datei in \textsc{OpenShift}-Cluster hoch.\\
		stelleBereit(für wen?, Datei) & Diese Methode schaltet die angegebene Datei für alle berechtigten Benutzerinnen auf dem Cluster frei.\\
		setzeSichtbarkeit(Sichtbarkeit) & Veröffentlicht die Komponenten, die durch die Konfigurationsdatei beschrieben wurden. Damit ist der Verteilungsvorgang abgeschlossen.\\
		abbruch(Fehlermeldung) & Bricht den Verteilungsprozess ab und gibt eine Fehlermeldung zurück.\\* \bottomrule
	
	\caption{Überblick über die verwendeten Methoden in Abbildung \vref{abb:prozessDeploymentCamunda}}\label{tab:methodenFunktion}\\
\end{longtable}