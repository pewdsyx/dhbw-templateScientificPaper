\chapter*{Lesehinweise}
Die folgenden Hinweise sollen das Lesen dieser Projektarbeit erleichtern und spezielle Formatierung definieren:

\begin{itemize}
	\item Im Sinne der Gleichberechtigung wird in dieser Arbeit entweder die weibliche Form oder die grammatikalisch korrekte Form \textit{\enquote{die/der Entwickler/-in}} verwendet werden. Die Kurzform (hier die weibliche Form) umschließt alle Geschlechter.
	\item Abbildungen, die mit dem Vermerk \textit{unternehmensintern} gekennzeichnet sind, unterliegen folgendem rechtlichen Hinweis: \enquote{Alle Rechte, einschließlich der Vervielfältigung, Veröffentlichung, Bearbeitung und Übersetzung bleiben der SV Informatik GmbH vorbehalten.}
	\item Produkt- oder Eigennamen werden in \textsc{Kapitälchen} gesetzt, wie beispielsweise \textsc{Node.js}.
	\item Hochgestellte Ziffern weisen auf Fußnoten am Seitenende hin.
	\item Die Zitation von Software-Dokumentationen gestaltet sich ohne Angabe der Seitenzahl im Original als schwierig. Im Folgenden wird der Stil zur Zitation solcher Dokumente beschrieben: \enquote{\cite[vgl.][Application\,$\rightarrow$\,Deployments]{red_hat_inc_okd_2019}}, wobei anstatt der Seitenzahl die Kapitel-Hierarchie mittels rechtsgerichtetem Pfeil (\enquote{$\rightarrow$}) illustriert wird. So kann die interessierte Leserin anhand der Kapitel in der jeweiligen Dokumentation zur Quelle blättern.
	\item Quellcode, der innerhalb des Fließtextes steht, wird folgendermaßen formatiert: \lstinline[language=Java]|System.out.println("Hello World!")|
	
	
\end{itemize}
 