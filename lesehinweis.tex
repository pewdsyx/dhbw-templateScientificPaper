\chapter*{Lesehinweise}
Die folgenden Hinweise sollen das Lesen dieser Projektarbeit erleichtern und spezielle Formatierung definieren:

\begin{itemize}
	\item Im Sinne der Gleichberechtigung wird in dieser Arbeit entweder die Form \textit{\enquote{die Entwickler*in}} oder die grammatikalisch korrekte Form \textit{\enquote{die/der Entwickler/-in}} verwendet werden. Bei der Kurzform mit der Sternnotation wird auf Grund der Lesbarkeit der weibliche Artikel benutzt.
	\item Abbildungen, die mit dem Vermerk \textit{unternehmensintern} gekennzeichnet sind, unterliegen folgendem rechtlichen Hinweis: \enquote{Alle Rechte, einschließlich der Vervielfältigung, Veröffentlichung, Bearbeitung und Übersetzung bleiben der SV Informatik GmbH vorbehalten.}
	\item Produkt- oder Eigennamen werden in \textsc{Kapitälchen} gesetzt, wie beispielsweise \textsc{Node.js}.
	\item Hochgestellte Ziffern weisen auf Fußnoten am Seitenende hin.
	
	
\end{itemize}
 