%
%   Prof. Dr. Julian Reichwald
%   auf Basis einer Vorlage von Prof. Dr. Jörg Baumgart
%   DHBW Mannheim
%
%
%	ACHTUNG: Für das Erstellen des Literaturverzeichnisses wird das modernere Paket biblatex
%			 in Kombination mit biber verwendet -- nicht mehr das ältere BibTex!
% 			 Bitte stellen Sie ggf. Ihre TeX-Umgebung
% 			 entsprechend ein (z.B. TeXStudio: Einstellungen --> Erzeugen --> Standard Bibliographieprogramm: biber)
%

\documentclass[
	12pt,
	BCOR=5mm,
	DIV=12,
	headinclude=on,
	footinclude=off,
	parskip=half,
	bibliography=totoc,
	listof=entryprefix,
	toc=listof,
	pointlessnumbers,
	plainfootsepline,
	headings=optiontohead]{scrreprt}


%	Konfigurationsdatei einziehen
\input{config}

\begin{document}

%% BITTE GEBEN SIE HIER DEN TITEL UND DIE AUTORIN / DEN AUTOR DER ARBEIT AN!
%% DIESE INFORMATIONEN _MÜSSEN_ GESETZT SEIN, UM TITELBLATT, ABSTRACT UND
%% EIGENSTÄNDIGKEITSERKLÄRUNG AUTOMATISCH ANZUPASSEN!

\TitelDerArbeit{Integration einer Container-Umgebung in einen automatisierten \enquote{Deployment}-Prozess und die Untersuchung ihrer Effekte auf diesen}
\AutorDerArbeit{Yves Torsten Staudenmaier}
\Firma{SV Informatik GmbH}
\Kurs{WWI17SEC}

\begin{titlepage}
\begin{minipage}{\textwidth}
		\vspace{-2cm}
		\noindent \includegraphics[scale=0.67]{img/logoeps/SVI-logo-claim.eps} \hfill   \includegraphics[scale=0.79]{img/logo.jpg}
\end{minipage}
\vspace{1em}
\sffamily
\begin{center}
	\textsf{\large{}Duale Hochschule Baden-W\"urttemberg\\[1.5mm] Mannheim}\\[2em]
	\textsf{\textbf{\Large{}Bachelorthesis}}\\[3mm]
	\textsf{\textbf{\DerTitelDerArbeit}} \\[1.5cm]
	\textsf{\textbf{\Large{}Studiengang Wirtschaftsinformatik}\\[3mm] \textsf{Studienrichtung Software Engineering}}
	
	\vspace{3em}
	\textsf{\Large{Sperrvermerk}}
\vfill

\begin{minipage}{\textwidth}

\begin{tabbing}
	Wissenschaftlicher Betreuer: \hspace{0.85cm}\=\kill
	Verfasser/-in: \> \DerAutorDerArbeit \\[1.5mm]
	Matrikelnummer: \> 123456789 \\[1.5mm]
	Firma: \> \DerNameDerFirma  \\[1.5mm]
	Abteilung: \> Abkürzung -- Name \\[1.5mm]
	Kurs: \> \DieKursbezeichnung \\[1.5mm]
	Studiengangsleiter/-in: \> Prof. Dr.-Ing. habil. Dennis Pfisterer \\[1.5mm]
	Wissenschaft. Betreuer/-in: \> Anna Musterfrau \\
	\> info@musterfrau.net \\
	\> +49 152 / 124 56789 \\[1.5mm]
	Firmenbetreuer/-in: \> Annalena Müller \\
	\> annalena.müller@unternehmen.de \\
	\> +49 123 / 113 0256 \\[1.5mm]
	Bearbeitungszeitraum: \> 17.02.--08.05.2020
\end{tabbing}
\end{minipage}

\end{center}

\end{titlepage}

\pagenumbering{Roman} % Römische Seitennummerierung
\normalfont

%--------------------------------
% Verzeichnisse - nicht benötige Verzeichnisse bitte auskommentieren / löschen.
%--------------------------------
\hyphenation{Open-Shift Ge-schäfts-sze-na-rios Ma-nage-ment Ma-nage-ment-Sys-tem} %weil es sonst nicht gut getrennt wird ...

%   Sperrvermerk
\chapter*{Sperrvermerk}
Der Inhalt dieser Arbeit darf weder als Ganzes noch in Auszügen Personen au"serhalb des Prüfungsprozesses und des Evaluationsverfahrens zugänglich gemacht werden, sofern keine anders lautende Genehmigung der Ausbildungsstätte vorliegt. Die Bachelorarbeit enthält unternehmensinterne Architektur- und Prozessmodellierungen und deren Dokumentation. Es ist zum Zeitpunkt der Anmeldung nicht sicher, ob interne Schnittstellen in der Anwendungslandschaft offengelegt werden.


\vspace{3cm}
\noindent Mannheim, 05.05.2020\hfill\rule{8.4cm}{.4pt}\par
\noindent\hfill Annalena Haus, Ausbildungsverantwortliche
\cleardoublepage



% Lesehinweis
\chapter*{Lesehinweise}
Die folgenden Hinweise sollen das Lesen dieser Projektarbeit erleichtern und spezielle Formatierung definieren:

\begin{itemize}
	\item Im Sinne der Gleichberechtigung wird in dieser Arbeit entweder die Form \textit{\enquote{die Entwickler*in}} oder die grammatikalisch korrekte Form \textit{\enquote{die/der Entwickler/-in}} verwendet werden. Bei der Kurzform mit der Sternnotation wird auf Grund der Lesbarkeit der weibliche Artikel benutzt.
	\item Produkt- oder Eigennamen werden in \textsc{Kapitälchen} gesetzt, wie beispielsweise \textsc{Node.js}.
	\item Hochgestellte Ziffern weisen auf Fußnoten am Seitenende hin.
	\item Der Quellcode des Prototyps liegt der Arbeit als CD bei. Außerdem kann dieser auf \textsc{Github} nach vorheriger Einladung eingesehen werden. Bitte schicken Sie eine E-Mail an \enquote{\href{mailto:yves-torsten.staudenmaier@sv-informatik.de}{yves-torsten.staudenmaier@sv-informatik.de}} mit der entsprechenden Anfrage. 
\end{itemize}
 

%	Kurzfassung
\chapter*{Kurzfassung}
\addcontentsline{toc}{chapter}{Abstract}
\begingroup
\begin{table}[h!]
\setlength\tabcolsep{0pt}
\begin{tabular}{p{3.7cm}p{11.7cm}}
Titel: & \DerTitelDerArbeit \\
Verfasser/-in: & \DerAutorDerArbeit \\
Kurs: & \DieKursbezeichnung \\
Ausbildungsstätte: & \DerNameDerFirma\\
\end{tabular}
\end{table}
\endgroup

%Überlege, ob ich den Header brauche mit den ganzen Infos? --> JA.
%Hier können Sie die Kurzfassung der Arbeit schreiben. 
Die \ac{SV} ist bestrebt ihre Versicherungsprozesse und den Kontakt mit den Kunden durch digitale Kanäle ständig zu verbessern. Die Digitalisierung ist ein wichtiger Bestandteil der \ac{SV}-Strategie: So ist sie Mitbegründerin der \enquote{id-fabrik}\footnote{Ein Start-up, das federführend Innovationen im Bereich der S-Finanzgruppe erzeugt.} und Mitglied im \enquote{InsurLab Germany}\autocite[vgl.][S.\,30]{sv_sparkassenversicherung_sv_2019}. Diese Anforderungen sollen durch eine Container-Plattform umgesetzt werden. Ziel der Arbeit ist, eine Betrachtung der technischen Umsetzbarkeit von einem Container-\enquote{Deployment} durchzuführen und ein Beispielprozess zu implementieren. Des Weiteren sollen die wirtschaftlichen- und sicherheits-relevanten Aspekte von diesem Prozess bzw. den Container-Anwendungen betrachtet werden. Methodisch sind Soll-Ist-Vergleiche, Anforderungsanalyse und Pseudocode verwendet. 
\par
Die Forschungsfrage eins beschäftigt sich mit der Frage wie Container-Anwendungen den Prozess des automatisierten \enquote{Deployments} unterstützen können. Diese Frage beschäftigt sich mit der Anforderungserhebung und der Implementierung eines angepassten Prozesses. Abschließend ist ein generischer Prozess entstanden, der Container-Anwendungen auf eine Orchestrierungsplattform zu verteilen. 
\par
Die Forschungsfrage zwei analysiert welchen wirtschaftlichen Vorteil die Container-Anwendungen für ein Unternehmen bieten. Dabei wird ein \enquote{Business case} aufgebaut, der die Investition in Container-Anwendungen analysiert und einordnet. Die Frage schließt mit dem Ergebnis ab, dass die formal korrekte Betrachtung eines \enquote{Business case} meist in der Unternehmensrealität nicht vollständig umgesetzt wird.
\par
Die Forschungsfrage drei beschäftigt sich mit der Analyse der Sicherheit von dem neuen Prozess. Dabei wird sich dem IT-Grundschutz-Kompendium des \ac{BSI} bedient, dass vorgefertigte Sicherheitsbausteine für IT-Komponenten beschreibt. Diese werden in einem Soll-Ist-Vergleich mit den bereits implementierten Sicherheitsanforderungen der \ac{SVI} verglichen. 


%	Inhaltsverzeichnis
\tableofcontents

%	Abbildungsverzeichnis
\listoffigures 

%	Tabellenverzeichnis
\listoftables

%	Listingsverzeichnis
\lstlistoflistings
 

% 	Algorithmenverzeichnis
\listofalgorithms

% 	Abkürzungsverzeichnis (siehe Datei acronyms.tex!)
\clearpage
\chapter*{Abkürzungsverzeichnis}	
\addcontentsline{toc}{chapter}{Abkürzungsverzeichnis}


\begin{acronym}[RDBMS]
	\acro{AWL}{Anwendungslandschaft}
	\acro{DHBW}{Dualen Hochschule Baden-Württemberg}
	\acro{BaFin}{Bundesanstalt für Finanzdienstleistungsaufsicht}
	\acro{BSI}{Bundesamt für Sicherheit in der Informationstechnik}
	\acro{VAIT}{Versicherungsaufsichtliche Anforderungen an die IT}
	\acro{IE2}{IE2 -- Deployment}
	\acro{IE}{IE -- Entwicklungs- und Betriebsunterstützung}
	\acro{CAB}{\enquote{Change Advisory Board}}
	\acro{ITIL}{Information Technology Infrastructure Library}
	\acro{SV}{SV SparkassenVersicherung}
	\acro{SVI}{SV Informatik GmbH}
	\acro{SVS}{SV Sachsen}
	\acro{i.d.R.}{in der Regel}

\end{acronym}

\ohead{Acronyms} % Neue Header-Definition

%--------------------------------
% Start des Textteils der Arbeit
%--------------------------------
\clearpage
\ihead{\chaptername~\thechapter} % Neue Header-Definition (inner header)
\ohead{\headmark} % Neue Header-Definition (outer header)
\pagenumbering{arabic}  % Arabische Seitenzahlen

%	Anleitungs-Datei anleitung.tex einziehen. Auf diese Weise sollten Sie versuchen, für jedes einzelne
% Kapitel eine eigene Datei anzulegen und mittels input-Kommando einzuziehen.
%\input{anleitung}

%-----------------
% Kapitel einbinden
%-----------------
%\vrefwarning %TODO: vor Abgabe weg machen!
% reset memory of all acronyms, so \ac will print out full name of acronym!
\acresetall 
\hyphenation{Bundes-anstalt} % because ngerman/babel doesn´t know it correctly... 
%TODO: need to fix hyphenation of "Bundesanstalt" in ac
\chapter{Einleitung}
\paragraph{Motivation der Arbeit}
irgendwas Originelles...
\begin{figure}[H]
	\centering
	\includegraphics[scale=0.33]{img/dilbertCloud.jpeg}
	\caption{Dilbert Comic zu \textsc{Kubernetes}}
	{\footnotesize Quelle: \cite{DilbertKubernetes}\par}
	{\footnotesize Redaktionelle Anmerkung: Abbildung nur als komprimiertes Format verfügbar (Qualitätseinbuße)}
\end{figure}

\paragraph{Problemstellung/-abgrenzung}

\paragraph{Zielstellung der Arbeit}
Folgende \textit{SMART}\footnote{\enquote{SMART-Regel, die eine einfache  Methode zum Operationalisieren von Zielen und auch zur Überprüfung der Güte von Zielen darstellt.} Quelle: \cite[][S.69]{dechange_projektmanagement_2020}}-formulierte Ziele sollen diese Bachelorthesis leiten, messbar machen und später als Anhaltspunkt zur Evaluierung des Erfolgs dienen:


\paragraph{Forschungsfragen/-design}
Die Forschungsfragen mit der sich diese Bachelorarbeit beschäftigen wird, sind eine direkte Konsequenz aus der Zielstellung und aus den unternehmensinternen Anforderungen an einen möglichen automatisierten Prozess. Dabei liegt der Fokus auf der Betrachtung beider Teildisziplinen der Wirtschaftsinformatik, nämlich der Informatik und der Wirtschaft -- jedoch wird der größere Teil dieser Arbeit einen informationstechnischen Fokus besitzen. Die folgende Aufzählung nennt die einzelnen Forschungsfragen, die im weiteren Verlauf ein gemeinsames Ergebnis erbringen werden. Dieses ist in Kapitel \vref{kritischeBetrachtung} zu finden.
\begin{enumerate}
	\item Wie können Container-Anwendungen den Prozess des automatisierten \enquote{Deployments}\footnote{\enquote{Software deployment may be considered to be a process consisting of a number of inter-related activities including the release of software at the end of the development cycle; the configuration of the software, the installation of software into the execution environment, and the activation of the software. It also includes post installation activities including the monitoring, deactivation, updating, reconfiguration, adaptation, redeploying and undeploying of the software.} (\cite{dearle_software_2007})} unterstützen?
	\item Welche wirtschaftlichen Vorteile hat der Einsatz von Container auf den Prozess des automatisierten \enquote{Deployments}?
	\item Welche besonderen sicherheitstechnischen Aspekte muss ein solcher Prozess im Bereich der Versicherung erfüllen?
\end{enumerate}
%TODO: Definition Technologie-Wertkette -> DevOps Handbuch O'Reilly
Die Forschungsfrage eins wird einen Ist-Zustand analysieren. Dieser enthält eine Prozessanalyse, eine identifizierte Technologie-Wertkette\footnote{Definition: <Defintion/>} sowie einen Anforderungskatalog der Entwicklungsabteilungen an den zu konzeptionierenden \enquote{Deployment}-Prozess für die Container-Anwendungen. Danach wird ein Konzept eines container-basierten, automatisierten \enquote{Deployment}-Prozesses erstellt, dabei wird die Methodologie und das eigentliche Konzept erläutert. Die Forschungsfrage eins schließt mit einem Teilergebnis ab. \par

Die Forschungsfrage zwei beschäftigt sich mit den wirtschaftlichen Vorteilen eines Einsatzes der Container auf den Prozess des automatisierten \enquote{Deployment}-Prozesses. Dabei werden die Erstellung eines \enquote{Business Case\footnote{engl. Geschäftsszenario}}, die Prüfung der Übereinstimmung der Ziele dieser Arbeit mit der Geschäftsstrategie der \ac{SVI} und mögliche Disharmonien dieser identifiziert. Außerdem entsteht eine Konzeption eines verbesserten Geschäftsszenarios, das die Kosteneinsparpotentiale und die Zielharmonisierung enthalten wird. Ein Ausblick schließt die Forschungsfrage zwei ab. \par

Die Forschungsfrage drei identifiziert sicherheitsrelevante Anforderungen, die nicht nur die funktionalen/nicht-funktionalen Anforderungen einer Anwendung betreffen, sondern auch die komplette \ac{AWL}. Dabei beeinflusst die Bundesanstalt für Finanzdienstleistungsaufsicht (\acs{BaFin}) und auch verschiedene \textsc{DIN/ISO}-Normen diese Anforderungen. Außerdem soll analysiert werden, wie bei der Beschaffung von \enquote{open source}- bzw. \enquote{closed source}-Anwendungen mögliche Schwachstellen identifiziert werden, die potenzielle Angriffsvektoren in der \ac{AWL} eröffnen würden, und wie mit diesen verfahren wird. Dabei soll versucht werden Rückschlüsse auf die Anwendung \textsc{OpenShift\footnote{\enquote{\textsc{OpenShift} is an open source container application platform by Red Hat based on the Kubernetes container orchestrator for enterprise app development and deployment.} Quelle: \cite[][]{red_hat_inc_openshift_2020}}} von \textsc{Red Hat\footnote{\enquote{Red Hat ist der weltweit führender Anbieter von Open Source-Lösungen, die auf verlässlichen und leistungsstarken Technologien in den Bereichen Cloud, Virtualisierung, Storage, Linux, Mobile und Middleware basieren. Darüber hinaus bieten wir Support-, Trainings- und Consulting-Services an, die mehrfach prämiert wurden.} Quelle: \cite[][]{red_hat_inc_red_2020}}} zu ziehen. Auch hier wird ein Teilergebnis diese Forschungsfrage abschließen


\paragraph{Einordnung der Abteilung in den Geschäftsprozess}
Die Abteilung \ac{IE2}, die sich im Bereich der Organisationseinheit \ac{IE} befindet, befasst sich in erster Linie mit dem Transport (\enquote{Deployment}) von Software-Artefakten der einzelnen Software-Produkte der \ac{SVI}. Diese werden für die \ac{SV} entwickelt, betrieben und gewartet. Zu den zentralen Aufgaben der Abteilung gehören die Planung, Durchführung und Überwachung der \enquote{Build/Deployment}-Prozesse auf den verschiedenen Serverumgebungen. Des Weiteren stellt \ac{IE2} die Einspielung von datenbank-relevanten Objekten sicher. Auch entwickelt sie die Bau- und Transportprozesse kontinuierlich weiter und passt diese an die sich ständig veränderten Anforderung der Entwicklungsabteilungen an. Von zentraler Bedeutung sind die Planung und Durchführung der Veröffentlichungen der neuen Versionen einer zu betreuenden Anwendung. Zu dieser Aufgabe gehören auch Aufbau und Bereitstellung der Systemtest-, Releasetest- und Produktionsumgebungen. Eine weitere zentrale Aufgabe, die nach der Organisationsumstrukturierung am 01.01.2020 in der Abteilung \ac{IE2} angesiedelt wurde, ist das Umgebungsmanagement. Die Aufgaben dieses Teilbereichs befasst sich mit folgenden Inhalten: Planung von Aktivitäten in der Produktionsumgebung, Planung und Koordination der Infrastruktur und Notfall-\enquote{Fix} der Produktion, der allgemeinen \enquote{Patch}-Planung; Beratung zur Erweiterung, Koordination und Planung von verschiedenen Testumgebungen. Außerdem ist das Umgebungsmanagement Teil des \ac{CAB}, das ein Gremium nach der Sammlung \ac{ITIL} darstellt. Dieses ist für die Freigabe von \enquote{Changes} verantwortlich und hat ständige, wie auch der Situation angepasste, Mitglieder. 

\paragraph{Aufbau der Arbeit}
In Kapitel \vref{ff1} \par
In Kapitel \vref{ff2} \par
In Kapitel \vref{ff3} \par
In Kapitel \vref{kritischeBetrachtung}

\chapter{Methodologie: Beschreibung des Vorgehens}\label{kap:methodology}
%Wie bin ich vorgegangen? Vorgehen auf Meta-Ebene beschreiben ...
% ANMERKUNG FÜR RITA: IN VERGANGENHEIT ZU VERFASSEN!!!
\paragraph{Forschungsfrage eins}
Die Anforderungsanalyse, die in dieser Forschungsfrage benutzt wurde, verwendet das Vorgehen von \cite{hull_requirements_2011} und \cite{partsch_requirements-engineering_2010}. Vor allem in \cite{hull_requirements_2011} wurde die methodische Vorgehensweise stark fokussiert, um eine hohe Qualität und Aussagekraft der Analyse zu gewährleisten. Deswegen nutzte diese Arbeit die dort illustrierte Methodik, da, erstens, es ein de-facto-Standard darstellt\footnote{Die Monographie wurde 1200 mal laut \textsc{Google Scholar} -- \cite{google_llc_google_2020} -- zitiert.} und, zweitens, das Vorgehen passend für die Problemstellung dieser Arbeit erschien. Die \enquote{stakeholder} wurden anhand des vorhandenen \enquote{Deployment}-Prozesses abgeleitet, d.\,h. es war beim aktuellen Prozess Projektbeteiligte in Form von Rollen definiert. Diese wurden für jede Ausprägung des Prozess mit bestimmen \enquote{stakeholder} gefüllt. Dadurch war der relevante Personenkreis aus den aktuellen Prozessen auf den zu implementierenden übertragen. Das \enquote{statement of needs} war durch den Fachbereich \enquote{Entwicklung} konkretisiert. Des Weiteren sollte ein Fragebogen die Wünsche der beteiligten Fachbereiche (\enquote{stakeholder}) aufnehmen, um daraus die präzisen Anforderungen abzuleiten. Dabei ward absichtlich eine offene Fragestellung an den Entwicklungs-Fachbereich (namentlich: \enquote{\textsc{Squad1}}) gesendet, da es keine Beeinflussung durch die Fragestellung geben sollte. Ziel war es, die wirklichen Wünsche des Fachbereichs herauszufinden. Aus den Wünschen wurden nach dem Muster von \cite[vgl.][S.\,28]{hull_requirements_2011} (siehe dazu auch Kapitel \vref{kap:methodikAnfAnalyse}) spezifische funktionale und nicht-funktionale Anforderungen formuliert. Als Formulierungshilfe, um die Verständlichkeit und die Messbarkeit zu waren, dienten die Regeln nach \cite{rupp_formulierungsregel_2020}. 
\par
Die Modellierung einer \enquote{\textsc{OpenShift}}-Labor-Umgebung diente dem Zweck der Anwendungsevaluierung, so wurde die Funktionsweise dieses System unter eingeschränkten Bedingungen getestet. Einschränkend wirkten die Limitierung auf eine \enquote{worker node}, die beschränkte Leistung der virtuellen Maschine, die Version von \textsc{OpenShift} und die Entkopplung von der internen \ac{AWL}. Hier wurde die \enquote{open source}-Variante genutzt, da so gewährleistet war, dass alle Tests und Verprobungen ohne Einschränkungen der Produktivumgebung durchgeführt wurden. Durch die Labor-Umgebung konnte die allgemeine Funktionsweise des Clusters getestet werden. Ziel dieser Umgebung war es, Tests zu ermöglichen und das Zusammenspiel der einzelnen Komponenten zu überprüfen. Aus diesem Grund wurde die Methodik der Labor-Umgebung gewählt, da sie einige Vorteile für diese Arbeit bot: Es konnte die Funktionsweise der Anwendungen getestet werden, die Konfigurationsdateien für das Deployment wurden unter realistischen Bedingungen erforscht und die Interaktion mit anderer Hilfssoftware ist evaluiert worden. Die Erkenntnisse wurden, soweit möglich, auf die \ac{AWL} übertragen. Problematisch war es, dass die kompletten Abhängigkeiten der einzelnen \enquote{\ac{AWL}}-Bestandteile in der Labor-Umgebung nicht vollständig modelliert wurden.
\par
% vielleicht Pseudo-Code für das Ergebnis nutzen, d.\,h. das Programm zur Generierung einer config-Datei
Die Mustererkennung beziehungsweise Analyse der Konfigurationsdatei, die die Entwicklungsabteilung bei der Arbeitstagung zum \ac{PoC} \textsc{OpenShift} vorstellte, wurde zur Analyse als zentraler Bestandteil herangezogen. Dabei wurde der Vergleich als Methodik benutzt, um die Unterschiede und deren Vorteile hervorzuheben. Das Vorgehen orientiert sich am Anwendungshersteller, der sich für einen Vergleich zwischen der angebotenen und der benötigten Funktionalität zur Findung der Lösung aussprach.\autocite[vgl.][Application\,$\rightarrow$\,Deployments]{red_hat_inc_okd_2019} Ziel der Analyse war es, die benötigten Funktionen herauszufinden und damit die Begründung für die Ausprägung der Konfigurationsdatei zu liefern. Abschließend wurde eine Vorlage entwickelt, die alle Anforderungen der einzelnen Beteiligten abdeckte. Der Algorithmus zur Generierung dieser, war in Pseudocode zur besseren Lesbarkeit dokumentiert. Diese Dokumentation ist der Nachvollziehbarkeit und dem Verständnis der Leserin geschuldet.
\par
% Prozessmodellierung
Die Prozessmodellierung des Container-\enquote{Deployments} nutze die Erkenntnisse aus der Anforderungsanalyse und der Labor-Umgebung, um einen möglichst stark an die Anforderungen angepassten Prozess zu erzeugen. Hierbei wurde das Sequenzdiagramm\footnote{Eine Art des Interaktionsdiagramms mit Fokus auf den Nachrichtenaustausch; spezifiziert in \cite[][S.\,595-599]{object_management_group_omg_unified_2017}.} der \ac{UML} in einer adaptierten Form zur Visualisierung des Prozesse genutzt. Folgende Anpassungen wurden vorgenommen: die \enquote{Lifeline} einer Instanz der Klasse stellte einen Prozessbeteiligten jeglicher Art dar. Sonst blieben die Semantik alle anderen Bestandteile unverändert. Das Sequenzdiagramm wurde gewählt, da es für die Beschreibung des Nachrichtenaustausches von verschieden Objekten benutzt wird. Dies war genau das darzustellende Ziel der Prozessmodellierung. Durch dieses Vorgehen wurde die Prozessmodellierung anwendungsunabhängig erstellt, um so dem Gedanken der Container\footnote{Hiermit ist das Desinteresse der \enquote{Deployment}-Abteilung an dem Inhalt des Containers gemeint, d.\,h. es sollte dem  \enquote{Deploy-Prozess} egal sein, was der Container enthält.} gerecht zu werden. 

\paragraph{Forschungsfrage zwei}
Die Geschäftsprozessanalyse nutzte die Methodik Teile der Ereignisgesteuerten Prozesskette (\acs{EPK}) nach \cite{scheer_objektorientierte_1997} und Teile des \ac{BPMN} nach \cite{object_management_group_omg_business_2011}. Ziel der verwendeten Methodik war es, einen besseren Überblick über den Gesamtprozess \enquote{Release} zu erhalten. Der Versuch die Optimierungspotentiale aufzudecken war durch die Verwendung der o.\,g. Methodik unterstützt. Es wurde absichtlich nicht das komplette Vorgehen des \enquote{Business case} nach \cite{brugger_it_2009} durchgeführt, da die \ac{SVI} sich nicht an dieses bei der Entscheidungsfindung hielt. So musste das Vorgehen zur \enquote{Business case}-Analyse adaptiert werden: Es entstand die Idee ein Ist-Soll-Vergleich durchzuführen. Dies entsprach nicht der wissenschaftlich anerkannten Methode der Kosten-Nutzen-Analyse, wie es der \enquote{Business case} vorschrieb, jedoch die praktikabelste Möglichkeit. Es waren keine Daten für die Berechnung von Investitionskennzahlen verfügbar, da die Entscheidung nicht auf Grund dieser Analyse vom Fachbereich getroffen wurde. Die Entscheidung Container-Anwendungen zu nutzen, war bedingt durch den Kauf der Software \textsc{Camunda}. 

\paragraph{Forschungsfrage drei}
Forschungsfrage drei
\chapter[Forschungsfrage 1]{Wie können Container-Anwendungen den Prozess des automatisierten \enquote{Deployments} unterstützen?} \label{ff1}
Kapiteleinleitung...

\section{Grundlagen: Definieren der Begrifflichkeiten zur Forschungsfrage eins}
Dieses Kapitel soll grundlegende Begrifflichkeiten, die im weiteren Verlauf dieser Arbeit verwendet werden, definieren, um so eine einheitliche Terminologie der Begriffe zu entwickeln. Dadurch wird ein gemeinsames Verständnis erzeugt.

\subsection{Methodik der Anforderungsanalyse}
Die Anforderungsanalyse leitet sich aus der Disziplin \enquote{Requirements-Engineering} ab, die verschiedene Bedeutungsvarianten besitzt -- dabei \enquote{[...] steht [es] einmal für alle konkreten Aktivitäten am Beginn einer Systementwicklung, die auf eine Präzisierung der Problemstellung abzielen. Ebenso steht es aber auch für eine ganze Teildisziplin im Grenzbereich zwischen Systems-Engineering, Informatik und Anwendungswissenschaften.}\autocite[][S.19]{partsch_requirements-engineering_2010}

\subsection{Cloud Computing}

\subsection{Container}

\subsection{\enquote{Deployment}} \label{defDeployment}

\section{Ist-Analyse des jetzigen \enquote{Deployment}-Prozesses}

\section{Konzeption eines container-basierten, automatisierten \enquote{Deployments}}
\chapter[Forschungsfrage 2]{Welche wirtschaftlichen Vorteile hat der Einsatz von Container auf den Prozess des automatisierten \enquote{Deployments}?} \label{ff2}
\chapter[Forschungsfrage 3]{Welche besonderen sicherheitstechnischen Aspekte muss ein solcher Prozess im Bereich der Versicherung erfüllen?} \label{ff3}
Dieses Kapitel beschreibt Grundlagen zur Sicherheit in der Informationstechnik. Die Grundlagen beschäftigen sich mit der ISO-Norm 27001, die eine vollständige Beschreibung enthält, wie Systeme gesichert werden können. Darauf aufbauend beschäftigt sich dieses Kapitel mit dem Kompendium des IT-Grundschutzes des \ac{BSI}, das mittels vorgefertigten Bausteinen eine praktische Anleitung zur Sicherung der IT-Systeme bietet. Außerdem wird der \ac{SVI}-Einkaufsprozess von \enquote{Open source} analysiert. Das Kapitel schließt mit einem Soll-Ist-Vergleich zwischen den Bausteinen des \ac{BSI} und der Umsetzung in der \ac{SVI} ab. Daraus resultiert das Teilergebnis der Forschungsfrage drei.
\par
Informations- und Kommunikationssysteme sind in der heutigen Gesellschaft von elementarer Bedeutung -- sie spielen eine immer größer werdende Rolle. Der Innovationsgrad in der Informationstechnik ist konstant hoch und deswegen sind folgende Bereiche ständiger Weiterentwicklung unterworfen: steigende Vernetzung der Bevölkerung, IT-Verbreitung und Durchdringung, Verschwinden der Netzgrenzen, kürzere Angriffszyklen auf wichtige Infrastruktur, höhere Interaktivität von Anwendungen und die Verantwortung der Benutzer eines IT-Systems.\autocite[vgl.][S.\,2f.]{bundesamt_fur_sicherheit_in_der_informationstechnik_bsi_it-grundschutz-kompendium_2020}

\section{Grundlagen: sicherheitstechnische Anforderungen}\label{kap:sicherheitstechnischeAnforderungen}
Informationen sind elementarer Bestandteil der heutigen Welt und diese sind von sehr hohem Wert für Unternehmen, Behörden und Privatpersonen. Die meisten Geschäftsprozesse, die im heutigen Prozessablauf einer Organisation verankert sind, funktionieren nicht ohne IT-Unterstützung. Somit ist die Informationstechnologie zentraler Bestandteil jedes Unternehmens. Deswegen ist ein zuverlässiges System mit entsprechender Soft- und Hardware unerlässlich. Es muss darauf geachtet werden, dass die Informationen, die auf diesem System verteilt sind, ausreichend gut geschützt sind, damit es nicht zu einer Bedrohungslage kommt. Unzureichend geschützte Systeme stellen ein sehr hohes Risiko dar. \enquote{Dabei ist ein vernünftiger Informationsschutz ebenso wie eine Grundsicherung der IT schon mit verhältnismäßig geringen Mitteln zu erreichen. Die verarbeiteten Daten und Informationen müssen adäquat geschützt, Sicherheitsmaßnahmen sorgfältig geplant, umgesetzt und kontrolliert werden. Hierbei ist es aber wichtig, sich nicht nur auf die Sicherheit von IT-Systemen zu konzentrieren, da Informationssicherheit ganzheitlich betrachtet werden muss. Sie	hängt auch stark von infrastrukturellen, organisatorischen und personellen Rahmenbedingungen ab.}\autocite[][S.\,1]{bundesamt_fur_sicherheit_in_der_informationstechnik_bsi_it-grundschutz-kompendium_2020} Die Mängel in der IT-Sicherheit führen meist zu folgenden drei Kategorien von Problemen\autocite[vgl.][S.\,1ff.]{bundesamt_fur_sicherheit_in_der_informationstechnik_bsi_it-grundschutz-kompendium_2020}: 

\begin{itemize}
	\item Verlust der Verfügbarkeit
	\item Verlust der Vertraulichkeit
	\item Verlust der Integrität
\end{itemize}

Der Verlust der Verfügbarkeit eines IT-Systems fällt sofort auf, da meist Aufgaben ohne diese Informationen nicht weitergeführt werden können. Meist tritt dies in dem Verlust der Funktionen eines Systems zutage. Die Vertraulichkeit von personenbezogenen Daten ist ein bestehendes Grundrecht jedes Bürgers beziehungsweise jedes Kunden. Dies ist in verschiedenen Gesetzen wie auch Verordnungen geregelt. Diese Daten müssen geschützt werden, da jedes Konkurrenzunternehmen Interesse an den Daten des Unternehmens hat. \enquote{Gefälschte oder verfälschte Daten können beispielsweise zu Fehlbuchungen, falschen Lieferungen oder fehlerhaften Produkten führen. Auch der Verlust der Authentizität (Echtheit und Überprüfbarkeit) hat, als ein Teilbereich der Integrität, eine hohe Bedeutung: Daten werden beispielsweise einer falschen Person zugeordnet. So können Zahlungsanweisungen oder Bestellungen zulasten einer dritten Person verarbeitet werden, ungesicherte digitale Willenserklärungen können falschen Personen zugerechnet werden, die digitale Identität wird	gefälscht.}\autocite[][S.\,1]{bundesamt_fur_sicherheit_in_der_informationstechnik_bsi_it-grundschutz-kompendium_2020}

\paragraph{\ac{ISMS}}\label{kap:ISMS}
Um ein \ac{ISMS} besser verstehen zu können, ist es wichtig, die Normenreihe des ISO-27001-Standards zu kennen. So bietet die ISO-Norm 27000 einen Überblick über ein solches System und definiert Begrifflichkeiten. Die zentrale Norm ist die ISO 27001, die die \ac{ISMS}-Anforderungen beschreibt.\autocite[vgl.][]{dindeutsches_institut_fur_normung_informationstechnik_2020} Dieser Norm sind die ISO-Standards 27002-27005, ISO 27007 und ISO 27008 untergeordnet, welche verschiedene Detailfragen zu in ISO 27001 genannten Konzepten definieren. Die Normen entstanden im britischen Institut für Standards, weswegen \enquote{[...] [es] gleichzeitig eine international anerkannte Zertifizierungsstelle für ISO 27001 [ist] und damit eine der Stellen, die befugt sind, Auditoren zu qualifizieren und einzusetzen, um die Übereinstimmung einer Organisation mit der ISO 27001 im Rahmen einer Zertifizierung zu überprüfen.}\autocite[vgl.][S.\,2]{kersten_it-sicherheitsmanagement_2020} Die ISO-Norm 27001 ist durch die abstrakte Beschreibung und ihren Aufbau auf jegliche Art von Organisationen (Behörden, Unternehmen, Vereine, \ac{NGOs} u.\,Ä.) anwendbar. Außerdem ist sie beliebig skalierbar und in jedem Land bzw. länderübergreifend nutzbar.\autocite[vgl.][S.\,4]{kersten_it-sicherheitsmanagement_2020} Die ISO-Norm 27000 definiert: \enquote{Ein Informationssicherheitsmanagementsystem (ISMS) umfasst Politik, Verfahren, Richtlinien und damit verbundene Ressourcen und Tätigkeiten,   die alle von einer Organisation gesteuert werden, um ihre Informationswerte zu  schützen. Ein ISMS ist ein systematisches Modell für die Einführung, die Umsetzung,  den Betrieb, die Überwachung, die Überprüfung, die Pflege und die Verbesserung der Informationssicherheit einer Organisation, um Geschäftsziele zu erreichen.}\autocite[][S.\,20]{dindeutsches_institut_fur_normung_informationstechnik_2019}
\par
Die ISO-Norm 27001 definiert in Kapitel vier bis zehn Anforderungen an ein Management-System der Informationssicherheit.\autocite[vgl.][S.\,6-16]{dindeutsches_institut_fur_normung_informationstechnik_2020} \enquote{Als Management-System für ein Thema X bezeichnet man allgemein alles, was eingesetzt wird, um die wesentlichen Ziele für das Thema X zu ermitteln, diese Ziele zu erreichen und ihre Aufrechterhaltung zu überwachen.}\autocite[][S.\,5]{kersten_it-sicherheitsmanagement_2020} Nachfolgend sind die typischen Aktivitäten genannt\autocite[][S.\,5]{kersten_it-sicherheitsmanagement_2020}:
\begin{itemize}
	\item Ziele in Form von Leitlinien zu formulieren,
	\item Risiken und Chancen für diese Ziele zu analysieren,
	\item Rollen bzw. Verantwortlichkeiten für bestimmte (Teil-)Ziele zu definieren,
	\item Methoden oder Verfahren zu deren Erreichung zu vermitteln,
	\item den vom Thema X Betroffenen besondere Regelwerke oder Richtlinien aufzugeben,
	\item Prozesse bzw. Abläufe und dafür erforderliche Maßnahmen zu planen und umzusetzen,
	\item Überprüfungen der Zielerreichung zu planen, durchzuführen und auszuwerten.
\end{itemize}
Ziel des \ac{ISMS} und damit der ISO-Norm 27001 ist es, für möglichen Prozess/Aktivitäten der Informationssicherheit ein einheitliches, standardisiertes System zu gestalten. Damit werden Aufwand- und Kosteneinsparungen erzeugt und die Akzeptanz eines solchen Systems gesteigert. Beispielsweise implementiert die ISO-Norm 9001 ein Qualitätsmanagementsystem\autocite{dindeutsches_institut_fur_normung_qualitatsmanagementsysteme_2020}, wobei die Architekturen der beiden Systeme kompatibel sind. Das \ac{ISMS} wird auf die gesamte Organisation angewendet. Die wichtigsten Aufgaben sind dabei die Formulierung von Sicherheitszielen, die Bestimmung des \enquote{Assets}\footnote{\enquote{Unter Assets wird alles verstanden, was für eine Organisation einen Wert darstellt. Dies können zunächst Grundstücke, Gebäude, Maschinen und Anlagen, Geschäftsprozesse sein – aber natürlich auch die sogenannten Information Assets (Informationswerte) wie Informationen/Daten, Systeme, Anwendungen, IT Services. Ergänzend kann man auch Soft Assets betrachten wie das Image oder die Kreditwürdigkeit einer Organisation.} Quelle: \cite[][S.\,8]{kersten_it-sicherheitsmanagement_2020}.}, die Risikobeurteilung und -behandlung und die kontinuierliche Verbesserung. Die Sicherheitsziele beschreiben die in Kapitel \vref{kap:sicherheitstechnischeAnforderungen} genannten drei Hauptziele der Informationssicherheit (Verfügbarkeit, Vertraulichkeit und Integrität). Das Kapitel der Leitlinien beschäftigt sich mit der Definition der Organisation, der Analyse und den Regeln auf verschiedenen Ebenen der Organisation. Der Prozess der kontinuierlichen Verbesserung implementiert das Modell des \enquote{Plan-Do-Check-Act}-Regelkreises.\footnote{Eine Abbildung dieses ist im Anhang \vref{abb:planDoCheckAct} zu sehen.} Eine akzeptierte Variante ist, den Regelkreis jährlich zu durchlaufen. Umso mehr Iterationen abgeschlossen sind, desto besser ist das \ac{ISMS}.\autocite[vgl.][S.\,16]{dindeutsches_institut_fur_normung_informationstechnik_2020} Der Anhang A der ISO-Norm 27001 definiert sogenannte \enquote{Controls}, die als Sicherheitsanforderung an die Organisation gestellt werden. Möchte die Organisation streng die ISO-Norm 27001 implementieren, so ist jede \enquote{Control} (es gibt 114) für jedes \enquote{Asset} aus der Inventarisierung umzusetzen. Um die Implementierung zu erleichtern, bietet die ISO-Norm 27002\autocite[vgl.][]{deutsches_institut_fur_normung_ev_informationstechnik_2017} Beispiele. Des Weiteren kann der IT-Grundschutz-Katalog des Bundesamtes für Sicherheit in der Informationstechnik (\acs{BSI})\footnote{Es gibt eine Tabelle, die die Implementierungsbeispiele des IT-Grundschutz zu den  \enquote{Controls} der ISO 27001 zuordnet. Quelle: \cite[][]{bundesamt_fur_sicherheit_in_der_informationstechnik_bsi_zuordnungstabelle_2018}.}, sowie das Wissen externer Beraterinnen genutzt werden, um mit den organisationseigenen Mitarbeitenden Maßnahmen zu entwerfen. Im Anhang \vref{tab:checklisteVorbereitungISMS} ist eine Checkliste abgebildet, die die Vorarbeiten der \ac{ISMS}-Einführung illustriert.
 
\paragraph{IT-Grundschutz-Katalog des \ac{BSI}}
Das IT-Grundschutz-Kompendium bildet mit den BSI-Standards 200-1, 200-2, 200-3 und dem \enquote{Leitfaden zur Basis-Absicherung} eine umfassende Beschreibung von Methoden, Anforderungen und Gefährdungen für die IT-Sicherheit. Dabei richten sie sich an Behörden und kleine, mittelständische und große Unternehmen.\autocite[vgl.][S.\,3]{bundesamt_fur_sicherheit_in_der_informationstechnik_bsi_it-grundschutz-kompendium_2020} Das IT-Grundschutz-Kompendium stellt dabei das Nachschlagewerk dar. Die BSI-Standards beschreiben, ähnlich zum ISO-Standard 27001, Themen, die das \ac{ISMS} betreffen. Der \enquote{Leitfaden zur Basis-Absicherung} ist die minimale Form der Implementierung von Sicherheitsanforderungen. Dieser kann für kleine Unternehmen schon ausreichend sein.\autocite[vgl.][S.\,5]{bundesamt_fur_sicherheit_in_der_informationstechnik_bsi_leitfaden_2017}
\par
\enquote{Im IT-Grundschutz-Kompendium werden standardisierte Sicherheitsanforderungen für typische Geschäftsprozesse, Anwendungen, IT-Systeme, Kommunikationsverbindungen und Räume in einzelnen Bausteinen beschrieben}.\autocite[][S.\,2]{bundesamt_fur_sicherheit_in_der_informationstechnik_bsi_it-grundschutz-kompendium_2020} Ziel dieses Schutzkompendiums ist es, einen für die Institutionen angepassten Schutz zu erreichen. Das Kompendium illustriert eine umfassende Methodik, die sich auf die organisatorische, personelle, infrastrukturelle und technische Sicherheit einer Institution bezieht. Es soll ein Sicherheitsniveau erreicht werden, das für die jeweilige Institution angemessen und mindestens ausreichend ist, um die relevanten Informationen zu schützen. 
Vorteil des Kompendiums ist das Baukastenprinzip, denn damit ist es möglich, sich leichter an die heterogene Umgebung der Informationstechnik anzupassen. Dies führt zu einer besser Planungsfähigkeit und Struktur der Maßnahmen.\autocite[vgl.][S.\,2]{bundesamt_fur_sicherheit_in_der_informationstechnik_bsi_it-grundschutz-kompendium_2020} Diese Bausteine bilden den Stand der Technik ab und können nach Bedarf kombiniert werden. Der besondere Vorteil dieses Prinzips ist die Reduzierung des Arbeitsaufwandes. Bei einer klassischen Risikoanalyse nach dem ISO-Standard 27001 u. a., wie im Kapitel \vref{kap:ISMS} beschrieben, muss für jedes \enquote{Asset} eine eigene Analyse durchgeführt werden: Dies kann entfallen, da das \ac{BSI} diese im Vorfeld abgeschlossen hat und die Ergebnisse in der jeweiligen Dokumentation des Bausteins zur Verfügung stellt. \enquote{Bei der IT-Grundschutz-Methodik reduziert sich die Analyse auf einen Soll-Ist-Vergleich zwischen den im IT-Grundschutz-Kompendium empfohlenen und den bereits umgesetzten Sicherheitsanforderungen. Die noch offenen Anforderungen zeigen die Sicherheitsdefizite auf, die es zu beheben gilt.}\autocite[][S.\,3]{bundesamt_fur_sicherheit_in_der_informationstechnik_bsi_it-grundschutz-kompendium_2020} Des Weiteren muss nur bei hohem Schutzbedarf (bspw. Schutz von systemkritischer Infrastruktur) eine Risikoanalyse für jedes \enquote{Asset} durchgeführt werden. Die Methodik der Risikoanalyse wird im BSI-Standard 200-3 \enquote{Risikoanalyse auf der Basis von IT-Grundschutz} weiter beschrieben. Ist ein Unternehmen bestrebt, eine Zertifizierung nach ISO 27001 zu erhalten, muss es die Basis- und Standard-Anforderungen des IT-Grundschutz-Kompendiums erfüllen. Darüber hinaus gibt es Anforderungen für einen erhöhten Schutzbedarf, die vom \ac{BSI} ausdrücklich empfohlen sind.\autocite[vgl.][S.\,3]{bundesamt_fur_sicherheit_in_der_informationstechnik_bsi_it-grundschutz-kompendium_2020} 
\par
Um einen Informationsverbund\footnote{\enquote{[...] ist die Gesamtheit von infrastrukturellen, organisatorischen, personellen und technischen Objekten zu verstehen, die der Aufgabenerfüllung in einem bestimmten Anwendungsbereich der Informationsverarbeitung dienen. Ein Informationsverbund kann dabei als Ausprägung die gesamte Institution oder	auch einzelne Bereiche, die durch organisatorische Strukturen (z. B. Abteilungen) oder gemeinsame Geschäftsprozesse bzw. Anwendungen (z. B. Personalinformationssystem) gegliedert sind, umfassen.} Quelle: \cite[][S.\,37]{bundesamt_fur_sicherheit_in_der_informationstechnik_bsi_it-grundschutz-kompendium_2020}.} nach dem IT-Grundschutz abzusichern, wird dieser mit den vorhandenen Bausteinen des Kompendiums nachgebildet. Es werden während dieses Prozesses alle IT-Systeme, Anwendungen und Prozesse erfasst und nach ihrem Schutzbedürfnis kategorisiert. Aus dieser Analyse wird ein IT-Grundschutz-Modell erstellt. Die Auswahl der passenden Komponenten oder Bausteine wird durch das Schichtenmodell\footnote{nicht zu verwechseln mit \ac{OSI}-Modell der Netzwerkprotokolle} (siehe Anhang \vref{abb:BSISchichtenmodell}) des IT-Grundschutz-Kompendiums vereinfacht. Um die Modellierung zu vereinfachen, werden die Bausteine jeder Schicht betrachtet, damit eine Entscheidung getroffen wird, in welchem Umfang diese zur Abbildung des Informationsverbundes nutzbar sind. Das Kompendium priorisiert die Bearbeitungsreihenfolge der Bausteine in drei Kategorien: \enquote{R1}, \enquote{R2} und \enquote{R3}. \enquote{R1}-Bausteine sollten vorrangig eingesetzt werden, da sie das Fundament des effektiven Sicherheitsprozesses bilden. Danach folgen Bausteine der beiden anderen Kategorien.

\section{Prozessbeschreibung: Beschaffung von \enquote{Open source}-Software}
In der \ac{SVI} gibt es, wie in den meisten anderen Unternehmen, eine prozessorientierte Vorgehensweise, um Software zu beschaffen. Die Beschaffung von Software orientiert sich an \ac{ITIL} Version 4, d.\,h., formal ist die Beschaffung von Software mit Hilfe eines \enquote{service requests}\footnote{a request from a user or a user's authorized representative that initiates a service action which has been agreed as a normal part of service delivery. Quelle: \cite[][S.\,195]{axelos_limited_itil_2019}.} zu beantragen. Für die Verteilung der Anwendung müssen danach mehrere \enquote{changes} eingereicht werden. Im weiteren Verlauf wird die \enquote{Open source}-Variante beleuchtet, da es sich bei den verwendeten Containern, um diese Variante handelt. Definitionsgemäß muss \enquote{Open source}-Software laut \cite{opensourceorg_open_2020} folgende Kriterien erfüllen: \enquote{Free redistribution, source code, derived works, integrity of the author's source Code, no discrimination against persons or groups, no discrimination against fields of endeavor, distribution of license, license must not be specific to a product, license must not restrict other software, license must be technology-neutral.} 
\par
Es gibt in der \ac{SVI} drei Prozesse, die sich in zwei Aspekten unterscheiden: Die Kosten und die Anforderungen, die an einen Prozess gestellt werden. Folgende Anfragen gibt es: die Beschaffungsanfrage, die \enquote{Freeware}-Beschaffung und die juristische Prüfung von Vertragsdokumenten oder Sachverhalten. Die Beschaffungsanfrage wird bei kostenpflichtiger Software gestellt. Da es in diesem Kapitel um kostenlose Software geht, wird auf die weitere Ausführung dieser Anfrage verzichtet. Der Prozess \enquote{Freeware}-Beschaffung wird laut den IT-Juristen der Abteilung \ac{IU11} kaum\footnote{$ n \leq 5, n \in \mathbb{N}_{0} $, gemessen p. a.} verwendet, denn die Fachbereiche\footnote{aus Sicht von \ac{IU11}} (die IT-Abteilungen) arbeiten zum jetzigen Zeitpunkt an dem Prozess vorbei -- sie übergehen diesen, obwohl es ihnen bekannt ist bzw. sein müsste. Folgende Probleme haben sich bei der Befragung der Fachbereiche herausgestellt: Die Anforderungen, die dieser Prozess an sie stellt, sind \enquote{nicht verhältnismäßig} gegenüber dem Nutzen. Die Fachbereiche wissen nicht, dass es einen solchen Prozess gibt oder ignorieren diesen. Die Anforderungen/Kriterien, die die Abteilung \ac{IU11} festgelegt hat, sind folgende: Es muss eine Produktverantwortliche definiert werden, es muss eine Architekturfreigabe von den zuständigen \enquote{Enterprise}-Architekten beantragt werden und es muss der genaue angedachte Verwendungszweck der einzukaufenden \enquote{Freeware}- bzw. \enquote{Open source}-Software definiert werden. Um den Ablauf des Prozesses besser verstehen zu können, zeigt Abbildung \vref{abb:pFW} ein adaptiertes Sequenzdiagramm, das den Fokus auf die Interaktion zwischen einzelnen Abteilungen legt.

\begin{figure}[H]
	\centering
	\includegraphics[scale=0.32]{img/prozessFreewareBeschaffung.pdf}
	\caption{(Adaptiertes) Sequenzdiagramm zur Beschaffung von \enquote{Open source}-Software}
	{\footnotesize Quelle: in Anlehnung an unternehmensinterne Prozessdokumentation \par \textit{unternehmensintern}}
	\label{abb:pFW}
\end{figure}

Die oben genannten Anforderungen erfüllen aus der Sicht der IT-Abteilung nicht die Kosten-Nutzen-Konformität. Aus rechtlicher Sicht ist das jedoch ein sehr hoch zu bewertendes Risiko, da es zu unmittelbaren juristischen Konsequenzen führen kann. Deswegen nutzt die IT-Abteilung meist den rechtlichen Prozess (juristische Prüfung von Vertragsdokumenten oder Sachverhalten), da dieser nicht die oben genannten Hürden enthält. Bei diesem wird der Verwendungszweck der Software erfragt und die geltenden Lizenzbedingungen durch \ac{IU11} geprüft. Jedoch ist davon auszugehen, dass eine offizielle Beschaffungsanfrage bei \enquote{Open source}-Software nur in wenigen\footnote{$ n \leq 10, n \in \mathbb{N}_{0} $, gemessen p. a.} Fällen gestellt wird. Begründet durch die Administrator-Berechtigung, die es Benutzern technisch erlaubt, auch ohne Restriktionen alles auf ihrem Computer zu installieren, kann keine numerische Aussage über die Dunkelziffer getroffen werden. Es bleibt nur die Hypothese der IT-Juristen der Abteilung \ac{IU11}, die weder falsifizierbar noch validierbar ist. 
\par
Ist die Software in der \ac{AWL} implementiert, gibt es noch eine Anwendung, \textsc{Nexus Lifecycle} von \textsc{sonatype}, die auf eventuelle Schwachstellen dieser benutzten Software prüft. \textsc{Nexus Lifecycle} ist eine Hilfsanwendung, die u. a. auch von \textsc{Creditreform} verwendet wird. Das Ziel dieses Produktes ist es, die gesamte Software-\enquote{Supply Chain} kontinuierlich zu bereinigen und sicher zu halten.\autocite[vgl.][]{sonatype_inc_nexus_2020} Aus dem Prüfbericht werden dann entsprechende Maßnahmen abgeleitet. Die erste ist die Software, in der die Schwachstelle gefunden wurde, als unsicher zu markieren und danach zu sperren. Die Entwicklungsabteilung muss versuchen, die Schwachstellen zu beseitigen. Problematisch ist es, wenn diese ignoriert werden. In letzter Konsequenz können der Betrieb und die Verteilung der Anwendung gestoppt werden. Dies führt zu massiven Problemen in der Produktion und somit verringert sich die vertragliche, mit dem Kunden vereinbarte, Verfügbarkeit der Systeme.

\section{Soll-Ist-Vergleich des in Forschungsfrage eins entworfenen Prozesses}
Die Abbildung \vref{abb:aufbauServerUmgebung} stellt den stark vereinfachten Aufbau der \textsc{OpenShift}-Umgebung im Rechenzentrum dar. Der Fokus liegt auf den einzelnen Komponenten. Diese können mittels eines Soll-Ist-Vergleichs zwischen den Anforderungen des Grundschutzes und den umgesetzten Vorkehrungen auf ihre Sicherheit überprüft werden. Dazu soll diese Abbildung \vref{abb:aufbauServerUmgebung} eine Übersicht der beteiligten Komponenten illustrieren. 

\begin{figure}[h!]
	\centering
	\includegraphics[scale=0.45]{img/aufbauOpenShiftServerUmgebung.pdf}
	\caption{Aufbau der \textsc{OpenShift}-Umgebung im Rechenzentrum}
	\label{abb:aufbauServerUmgebung}
	{\footnotesize Quelle: in Anlehnung an unternehmensinterne Dokumentation\par}
	{\footnotesize \textit{unternehmensintern}}
\end{figure}

Es werden folgende Komponenten identifiziert, die einer näheren Betrachtung und Bewertung durch das IT-Grundschutz-Kompendium des \ac{BSI} bedürfen: der \enquote{bare metal}-Server, der virtualisierte Server, das \textsc{Linux}-basierte Betriebssystem, die \textsc{OpenShift}-Anwendung, das Netzwerk beziehungsweise die demilitarisierte Zone\footnote{Anmerkung: Im Sinne der Informatik.} und das Rechenzentrum (der Vollständigkeit geschuldet). Die Bausteine (\enquote{Prozess-Bausteine}) werden mit Abkürzungen beschrieben.\autocite[vgl.][S.\,2-4]{bundesamt_fur_sicherheit_in_der_informationstechnik_bsi_it-grundschutz-kompendium_2020}\footnote{Außerdem ist das Schichtenmodell im Anhang \vref{abb:BSISchichtenmodell} dargestellt.} Die nachfolgende Betrachtung ist lediglich ein Überblick und müsste für eine Zertifizierung nach ISO 27001 detailreicher durchgeführt werden.
\par
Die Zuordnung der identifizierten Komponenten zu den Bausteinen des Grundschutzes ist der nächste wichtige Schritt, um den Soll-Ist-Vergleich durchzuführen. Die Tabelle \vref{tab:zuordnungKompBau} beschreibt diese Zuordnung. Im weiteren Verlauf wird die Kurzform der Komponenten-Baustein-Zuordnung mit der mathematisch korrekten Schreibweise $ Komponente\ k \mapsto Baustein\ b $ dargestellt.
\par

\begin{longtable}{@{}lp{8.0cm}@{}}
	\toprule[1.5pt]
	\textbf{Komponente} & \textbf{Bausteine des IT-Grundschutzes} \\* \midrule
	\endfirsthead
	%
	\multicolumn{2}{c}%
	{{\bfseries Tabelle \thetable\ von vorheriger Seite fortgeführt.}} \\
	\toprule
	\textbf{Komponente} & \textbf{Bausteine des IT-Grundschutzes} \\* \midrule
	\endhead
	%
	\bottomrule
	\endfoot
	%
	\endlastfoot
	%
	% below rules with content
	\enquote{Bare metal}-Server & SYS.1.1 Server\\
	Virtualisierte Server & SYS.1.5 Virtualisierung \\
	\textsc{Linux}-basierte Betriebssystem & SYS.1.3 Server unter Linux und Unix\\
	\textsc{OpenShift}-Anwendung & APP.3.1 Web-Anwendungen\\
	\pagebreak
	Netzwerk bzw. demilitarisierte Zone & NET.1 Netze und NET.3.2 \enquote{Firewall}\\
	Rechenzentrum & INF.2 Rechenzentrum sowie Serverräume\\* 
	
	\bottomrule[1.5pt]
	
	\caption{Zuordnung der Komponenten zu den Bausteinen}\label{tab:zuordnungKompBau}\\
\end{longtable}

Das \ac{ISMS} ist in der \ac{SVI} über den IT-Grundschutz implementiert. Die \ac{SV} verfügt über hochsensible, persönliche Kundendaten, wie z.\,B. Gesundheitsdaten. Diese sind hoch schutzbedürftig, da der Verlust, die Offenlegung oder die Verbreitung ein Vergehen nach dem Strafgesetzbuch\footnote{Das sind die einzigen Daten, die bei nicht-rechtskonformer Nutzung eine Freiheitsstrafe auslösen können.} darstellt. Für die Bereiche, in denen mit diesen Daten gearbeitet wird, konnten die vorgefertigten Bausteine nicht benutzt werden. Diese Bereiche sind einer vollumfänglichen Risikoanalyse nach dem ISO-Standard 27001 unterzogen worden. Da die Anwendung \textsc{Camunda}, für die der \enquote{Deployment}-Prozess initial erstellt wurde, keine Gesundheitsdaten verarbeiten wird, ist die Verwendung der Bausteine des \ac{BSI} laut der Empfehlung des \ac{BSI} ausreichend. Jedoch ist der \enquote{Deployment}-Prozess generisch angelegt und kann auch andere Container-Anwendungen verteilen. Die vorliegende Analyse bezieht sich dennoch auf eine Umgebung, die mittleren Schutzes bedarf, d.\,h., die Bausteine können verwendet werden. Dies ist begründet durch die Aussage der \enquote{Enterprise}-Architekten, dass ein Umzug der Bestandsführungssysteme (sie enthalten die hochsensiblen Daten) in Container-Anwendungen in den nächsten fünf Jahren nicht geplant sei. 
\par
\textbf{Komponente \enquote{Bare metal}-Server}:\label{sec:bareMetalServer}
\par
Diese wird dem Baustein \textit{SYS.1.1 Server} zu geordnet. Per Definition des \ac{BSI} ist ein allgemeiner Server folgendes Konstrukt: \enquote{[...] werden IT-Systeme mit einem beliebigen Betriebssystem bezeichnet, die Benutzern und anderen IT-Systemen Dienste bereitstellen. Diese Dienste können Basisdienste für das lokale oder externe Netz sein, aber auch den E-Mail-Austausch ermöglichen oder Datenbanken und Druckerdienste anbieten.}\autocite[][S.\,461]{bundesamt_fur_sicherheit_in_der_informationstechnik_bsi_it-grundschutz-kompendium_2020} Die identifizierte Gefährdungslage der \ac{BSI} sind Software-Schwachstellen und -Fehler, Datenverlust, \enquote{Denial of Service}-Angriffe, Überlastung der Server und Bereitstellung unnötiger Komponenten und Applikationen.\autocite[vgl.][S.\,461-462]{bundesamt_fur_sicherheit_in_der_informationstechnik_bsi_it-grundschutz-kompendium_2020} Diese Bedrohungen sind von der \ac{SVI} und der Rechenzentrumsbetreiberin erkannt. Diese Betreiberin steht in der Verantwortung, die Anforderungen des \ac{BSI} umzusetzen, da die \ac{SVI} das Rechenzentrumsmanagement an eine Dienstleisterin übergeben hat. Somit muss diese die Anforderungen implementieren. Jedoch kontrolliert die \ac{SVI} gemeinsam mit der \ac{SV} die Betreiberin durch ein internes jährliches Audit. Dies ist im Rahmenvertrag mit dem Unternehmen \textsc{Cancom} vereinbart. Auch der ständige Informationsaustausch ist eine Rahmenbedingung des Vertrages. Durch die Abtretung der Umsetzungsverantwortung begründet wird dieser Baustein durch diese Arbeit nicht näher beleuchtet. Die Maßnahmen des \ac{BSI} sind laut Geschäftsführung vollumfänglich implementiert.
\par
\textbf{Komponente Virtualisierte Server}:
\par
Für die Bewertung dieser Server wird laut \ac{BSI} folgender Baustein benutzt: $VServer \mapsto SYS.1.5\ Virtualisierung$. Die \ac{BSI} definiert: \enquote{Bei der Virtualisierung von IT-Systemen werden ein oder mehrere virtuelle IT-Systeme auf einem physischen IT-System ausgeführt. Ein solches physisches IT-System wird als „Virtualisierungsserver“ [sic!] bezeichnet. Mehrere Virtualisierungsserver können zu einer virtuellen Infrastruktur zusammengefasst werden. Darin können die Virtualisierungsserver selbst und die auf ihnen betriebenen virtuellen IT-Systeme gemeinsam verwaltet werden.}\autocite[][S.\,485]{bundesamt_fur_sicherheit_in_der_informationstechnik_bsi_it-grundschutz-kompendium_2020} Die identifizierten Gefährdungslagen sind fehlerhafte Planung oder Konfiguration der Virtualisierung, unzureichende Ressourcen für die IT-Systeme, Informationsabfluss oder Ressourcenengpass durch \enquote{snapshots}\footnote{Speicherung eins gewissen (vollumfänglichen) Zustands der virtuellen Maschine.}, Ausfall des Verwaltungsservers, missbräuchliche Nutzung von sogenannten Gastwerkzeugen\footnote{Damit können Laufwerke, Funktionen und Programme auf dem Gastsystem eingespielt werden.} und die Kompromittierung der Virtualisierungssoftware.\autocite[vgl.][S.\,486-487]{bundesamt_fur_sicherheit_in_der_informationstechnik_bsi_it-grundschutz-kompendium_2020} Diese Server unterliegen auch dem Management der Dienstleisterin der \ac{SVI}. Deswegen ist die Begründung die gleiche, wie in Kapitel \vref{sec:bareMetalServer} Absatz \enquote{Komponente \enquote{Bare metal}-Server}. 
\par
\textbf{Komponente \textsc{Linux}\footnote{Formal nicht korrekt, da es \textsc{Unix}- und \textsc{Linux}-Systeme gibt. Hier wird \textsc{Linux} als Sammelbegriff für beide Gattung verwendet.}-basierte Betriebssysteme}:
\par
Diese können durch die Anforderungen des Bausteins \textit{SYS.1.3 Server unter Linux und Unix} abgesichert werden. Die Definition des \textit{SYS.1.3} lautet: \enquote{Auf Server-Systemen werden häufig die Betriebssysteme Linux oder Unix eingesetzt. Beispiele für klassische UnixSysteme [sic!] sind die BSD-Reihe (FreeBSD, OpenBSD und NetBSD), Solaris und AIX. Linux bezeichnet hingegen kein klassisches Unix, sondern ist ein funktionelles Unix-System. Das heißt, dass der Linux-Kernel nicht auf dem ursprünglichen Quelltext basiert, aus dem sich die verschiedenen Unix-Derivate entwickelt haben. In diesem Baustein werden alle Betriebssysteme der Unix-Familie betrachtet, also auch Linux als funktionelles Unix-System. Da sich die	Konfiguration und der Betrieb von Linux- und Unix-Servern ähneln, werden in diesem Baustein Linux und Unix sprachlich als „Unix-Server“ bzw. „unixartig“ zusammengefasst.}\autocite[][S.\,480]{bundesamt_fur_sicherheit_in_der_informationstechnik_bsi_it-grundschutz-kompendium_2020} Die Gefährdungslagen sind Ausspähen von System- und Benutzerinformationen, Ausnutzbarkeit der Skript-Umgebung, dynamisches Laden von gemeinsam genutzten Bibliotheken und Software aus Drittquellen.\autocite[vgl.][S.\,480]{bundesamt_fur_sicherheit_in_der_informationstechnik_bsi_it-grundschutz-kompendium_2020} Diese Systeme sind in der vollständigen Verantwortung der \ac{SVI} und müssen deswegen durch die Implementierung von Sicherheitsmaßnahmen geschützt werden. Das \ac{BSI} definiert die grundsätzliche Zuständigkeit für die Systeme als Abteilung Betrieb. Des Weiteren sind folgende Anforderungen an diese IT-Systeme beschrieben: Authentisierung von Administratorinnen und Benutzerinnen, sorgfältige Vergabe von Benutzerinnenkennungen, kein automatisches Einbinden von Wechsellaufwerken, Schutz vor Ausnutzung von Schwachstellen in Anwendungen, sichere Installation von Software-Paketen, zentrale Verwaltung von Benutzerinnen und Gruppen, verschlüsselter Zugriff auf die Skript-Umgebung, Absicherung des Startvorgangs, der Einsatz von Sicherungsmaßnahmen des Dateisystems und zusätzlicher Schutz bei erhöhtem Sicherheitsbedarf (abgekürzt, da nicht relevant für die \textsc{Linux}-Systeme der \ac{SVI})\autocite[vgl.][S.\,480-482]{bundesamt_fur_sicherheit_in_der_informationstechnik_bsi_it-grundschutz-kompendium_2020}. Die \ac{SVI} setzt die oben genannten Maßnahmen um. Die Authentifizierung von Administratorinnen und Benutzerinnen erfolgt durch die Anmeldung an dem Server. Benutzerinnenkonten müssen immer über einen internen Beauftragungsprozess beantragt werden. Dadurch kann keine Benutzerin ihre Rechte auf einem System ohne Freigabe ändern bzw. erweitern. Bei der Beauftragung müssen alle zusätzlichen Privilegien begründet werden, z.\,B. das Recht ein Skript oder ein Programm als Administratorin auszuführen. Jedes Benutzerinnenkonto ist personalisiert anzulegen, da somit die Nachverfolgbarkeit gewährleistet ist. Technische Benutzerinnen sind auf den IT-Systemen abzuschaffen und durch personalisierte Versionen zu ersetzen. Diese Anforderung begründet die \ac{SVI} durch die Sicherheitsvorschriften. Durch den Prozess wird auch eine übermäßige, unkontrollierte Ausgabe von Konten auf einem System verhindert, wodurch zu jedem Zeitpunkt bekannt ist, welche Personen Zugriff auf das System mit welchen Rechten haben. Wechsellaufwerke können auf den Systemen nicht eingebunden werden. Die Kernel der Systeme der \ac{SVI} sind durch die für Finanzdienstleistung vorgeschrieben Regularien geschützt, so ist die Verwendung von Drittanbieterinnen-Anwendungen nicht möglich. Auch sind alle Sicherheits-\enquote{Updates} einzuspielen. Die Installation von Software erfolgt bei der \ac{SVI} nicht durch die Benutzerin der Anwendung, sondern sie beauftragt die Installation mittels \enquote{service request}. Dieser Prozess implementiert Sicherheitsvorkehrungen. So ist die Installation immer von mindestens einer anderen nicht beteiligten Person kontrolliert. Die Dateien und Zustände der Systeme werden durch einen ständigen Zyklus gesichert. Diese Datensicherung ist unabhängig von der \ac{AWL} betrieben, d.\,h., bei einem Virusbefall sind die Sicherungen physisch nicht erreichbar und somit geschützt. Wichtig ist es, dass die Maßnahmen ständig auf ihre Wirksamkeit überprüft werden. Die \ac{SVI} hat dazu interne Sicherheits-Audits implementiert, die im sechsmonatigem Rhythmus überprüfen, ob alle Sicherheitsstandards eingehalten werden. Die Betriebsabteilungen sind angehalten, Sicherheits-\enquote{Updates} direkt nach Bekanntgabe der Herstellerinnen über deren Verfügbarkeit in die Systeme einzuspielen.
\par
\textbf{Komponente} \textsc{OpenShift}\textbf{-Anwendung:}
\par
Diese Komponente wird durch die Anforderungen des Bausteins \textit{APP.3.1 Web-Anwen-dungen} teilweise beschrieben. Die Einschränkung bezieht sich auf die Nutzung des Kunden über eine Webanbindung mittels des Protokolls \ac{HTTPS}. Die Absicherung der Webdienste wird nachfolgend betrachet. Die Definition des Begriffes \textit{APP.3.1} lautet: \enquote{Webanwendungen stellen Funktionen und dynamische Inhalte zur Verfügung. Sie nutzen dazu das Internetprotokoll HTTP (Hypertext Transfer Protocol) bzw. HTTPS. Bei HTTPS wird die Verbindung durch die Protokolle SSL (Secure Socket Layer) oder TLS (Transport Layer Security) verschlüsselt und zusätzlich gesichert. Webanwendungen erstellen auf einem Server Dokumente und Benutzeroberflächen, z. B. Eingabemasken, und liefern diese an entsprechende Clientprogramme, wie etwa Webbrowser. Webanwendungen werden gewöhnlich auf der Grundlage von Frameworks entwickelt. Diese stellen ein Rahmenwerk für häufig wiederkehrende Aufgaben zur Verfügung, z.\,B. für Sicherheitskomponenten.}\autocite[][S.\,383]{bundesamt_fur_sicherheit_in_der_informationstechnik_bsi_it-grundschutz-kompendium_2020} Die Sicherheit wird durch Anforderungen des \ac{BSI} umgesetzt. Diese sind: Authentifizierung der Entwicklerinnen, Zugriffskontrolle, Protokollierung sicherheitsrelevanter Ereignisse, Schutz vor unerlaubter Nutzung, Schutz von vertraulichen Daten, Architekturanforderungen, sichere Anbindung von Hintergrundsystemen und Konfiguration der Webanwendung.\autocite[vgl.][S.\,385-389]{bundesamt_fur_sicherheit_in_der_informationstechnik_bsi_it-grundschutz-kompendium_2020} Der Baustein \textit{APP.3.1} ist initial nicht für eine \enquote{Cloud}-Anwendung, wie es \textsc{OpenShift} ist, bestimmt. Das \ac{BSI} beschreibt in diesem Baustein den Webserver, der \textsc{HTML}- oder \textsc{PHP}-Seiten darstellt. Jedoch bietet das \ac{BSI} den Baustein \textit{OPS.2.2 Cloud-Nutzung} an, der sich auf die allgemeine Nutzung von \enquote{Cloud}-Diensten bezieht. Dieser Baustein definiert \enquote{Cloud Computing} als \enquote{[Cloud Computing] bezeichnet das dynamisch an den Bedarf angepasste Anbieten, Nutzen und Abrechnen von IT-Dienstleistungen über ein Netz. Angebot und Nutzung dieser Dienstleistungen erfolgen dabei ausschließlich über definierte technische Schnittstellen und Protokolle. Die Spannbreite der im Rahmen von Cloud Computing angebotenen Dienstleistungen umfasst das komplette Spektrum der Informationstechnik und beinhaltet unter anderem Infrastruktur (z. B. Rechenleistung, Speicherplatz), Plattformen und Software.}\autocite[][S.\,265]{bundesamt_fur_sicherheit_in_der_informationstechnik_bsi_it-grundschutz-kompendium_2020} Dieser Baustein spiegelt den Vorteil der Plattform \textsc{OpenShift} wider, die Anwendungen dynamisch skaliert. Die Sicherheitsvorkehrungen dieses Bausteines sind die Erstellung einer \enquote{Cloud}-Nutzungsstrategie und von Sicherheitsrichtlinien, die Festlegung von Verantwortlichkeiten und Schnittstellen, die Erstellung eines Notfallkonzepts und Bedingungen, die einen Ausstieg der \enquote{Cloud}-Nutzung ermöglichen.\autocite[vgl.][S.\,268-271]{bundesamt_fur_sicherheit_in_der_informationstechnik_bsi_it-grundschutz-kompendium_2020} Die \ac{SVI} betreibt die \textsc{OpenShift}- und damit die \enquote{Cloud}-Umgebung im  eigenen Rechenzentrum, das von ihrer Dienstleisterin betrieben wird. Jedoch ist die \ac{SVI} für die Sicherung der Anwendungen selbst verantwortlich. Die Nutzungsstrategie beschränkt sich zum Zeitpunkt der Erstellung der Arbeit auf die Verprobung der Anwendung \textsc{Camunda} in einer Labor-Umgebung. Die Nutzungsstrategie für das Produktivsystem wird von den \enquote{Enterprise}-Architekten in Zusammenarbeit mit der IT-Sicherheitsabteilung ausgearbeitet. Die Verantwortlichkeiten sind strikt aufgeteilt: Die Betriebsabteilung betreut das Management und die administrativen Aufgaben; die Abteilung \ac{IE2} entwickelt, betreibt, prüft und dokumentiert die \enquote{Deployments} und die Entwicklungsabteilungen bzw. -teams erstellen die Container-Anwendungen zur Übergabe an \ac{IE2}. Die initialen Sicherheitsrichtlinien sind von den Entwicklungsrichtlinien abgeleitet. Diese schreiben beispielsweise vor, bei Webanwendungen alle \enquote{Ports} zu blocken bis auf die notwendigen \enquote{Ports}. Die Schnittstellen werden in drei aufgeteilt: Übergabe der entwickelten Anwendung an \ac{IE2} durch die Entwicklungsabteilungen, die Beauftragung von administrativen Aufgaben durch, erstens, die Entwicklungsteams oder, zweitens, durch die Abteilung \ac{IE2}. Das Notfallkonzept schreibt u. a. vor, dass die \textsc{OpenShift}-Umgebung ein vollständiges Parallel-System haben muss, das im Notfall einspringt. Dies wurde bei der Kaufentscheidung berücksichtigt. Der Ausstieg aus Systemen wird in der \ac{SVI} bei eigenem Betrieb im Rechenzentrum nur geplant, wenn die Anforderung der Ablösung besteht.
\par
\textbf{Komponente Netzwerk bzw. demilitarisierte Zone}:
\par
Die Sicherheit dieses Bausteins wird durch die Anforderung des \textit{NET.1 Netze und NET.3.2 \enquote{Firewall}} beschrieben. Die Definition des Begriffs \textit{NET.1} lautet: \enquote{Die meisten Institutionen benötigen heute für ihren Geschäftsbetrieb und für die Erfüllung ihrer Fachaufgaben Datennetze, über die beispielsweise Informationen und Daten ausgetauscht sowie verteilte Anwendungen realisiert	werden. In solche Netze werden nicht nur herkömmliche Endgeräte, Partner-Institutionen und das Internet angeschlossen. Sie integrieren zunehmend auch mobile Endgeräte und Elemente, die dem Internet of Things (IoT) zugerechnet werden. Darüber hinaus werden über Datennetze vermehrt auch Cloud-Dienste sowie Dienste für Unified	Communication and Collaboration (UCC) [sic!] genutzt. Die Vorteile, die sich dadurch ergeben, sind unbestritten.}\autocite[][S.\,669]{bundesamt_fur_sicherheit_in_der_informationstechnik_bsi_it-grundschutz-kompendium_2020} Die \enquote{Firewall} beschreibt das \ac{BSI} so: \enquote{Eine Firewall ist ein System aus soft- und hardwaretechnischen [sic!] Komponenten, das dazu eingesetzt wird, IP-basierte Datennetze sicher zu koppeln. Dazu wird mithilfe einer Firewall-Struktur der technisch mögliche Informationsfluss auf die in einer Sicherheitsrichtlinie als vorher sicher definierte Kommunikation eingeschränkt. Sicher bedeutet hierbei, dass ausschließlich die erwünschten Zugriffe oder Datenströme zwischen verschiedenen Netzen zugelassen werden.}\autocite[][S.\,711]{bundesamt_fur_sicherheit_in_der_informationstechnik_bsi_it-grundschutz-kompendium_2020} Das Netzwerk und die demilitarisierte Zone befinden sich, wie auch die Server-Komponenten und nachfolgend das Rechenzentrum, in der vollumfänglichen Verantwortung der \ac{SVI}-Dienstleisterin. Die Regelungen des Rahmenvertrages mit dem Unternehmen \textsc{Cancom} gelten auch hier. Die Audits aus den Jahren 2018 und 2019 haben ergeben, dass es keine relevanten Sicherheitsvorfälle gab und das vereinbarte Sicherheitslevel zu jedem Zeitpunkt eingehalten war. 
\par
\textbf{Komponente Rechenzentrum}:
\par
Diese ist durch den Baustein \textit{INF.2 Rechenzentrum sowie Serverräume} beschrieben. Eine der Definitionen des \ac{BSI} lautet: \enquote{Wird die IT der Institution innerhalb eines Gebäudes oder einer Liegenschaft verteilt in mehreren Bereichen betrieben und sind diese Bereiche untereinander und zu den IT-Benutzern hin durch hauseigene LAN-Verbindungen angeschlossen, ist mindestens der funktional bedeutendste dieser Bereiche als RZ zu behandeln. Des Weiteren sind Bereiche, von deren ordnungsgemäßem Betrieb 50\,\% und mehr Nutzer abhängig sind oder aus denen heraus 50\,\% und mehr an Diensten und Daten (gemessen an der Gesamtheit der Bereiche) bereitgestellt werden, als RZ zu behandeln.}\autocite[][S.\,755]{bundesamt_fur_sicherheit_in_der_informationstechnik_bsi_it-grundschutz-kompendium_2020} Diese Definition trifft auf die Lage des Rechenzentrums der \ac{SVI} zu, denn sie ist an zwei Orten im Stuttgarter Raum verteilt und besteht aus einer Serverlandschaft. Der Betrieb, die Wartung und das Management (zu großen Teilen) werden durch die Dienstleisterin der \ac{SVI} durchgeführt. Die Verantwortung der Erfüllung aller sicherheitsrelevanter Bestimmungen und Maßnahmen obliegt der Betreiberin. Diese sind in einem Rahmenvertrag mit der Dienstleisterin \textsc{Cancom} vereinbart. Begründet durch diese Tatsache sieht diese Arbeit von der weiteren Betrachtung der Komponente Rechenzentrum ab.

\section{Ergebnis der Forschungsfrage drei}
Abschließend ist hervorzuheben, dass dieses Kapitel einen Überblick über die Sicherheitskonzepte der \ac{SVI} darstellt. Eine vollständige Risikoanalyse übersteigt den Umfang dieser Arbeit, da der Fokus der Arbeit auf der Forschungsfrage eins (Kapitel \vref{ff1}) liegt. 

\paragraph{Abschließende Sicherheitsprüfung} Die Sicherheitsprüfung bzw. der Vergleich ist nicht abgeschlossen und genügt den Anforderungen nicht, die im Grundlagenteil dieses Kapitels (vgl. Kapitel \vref{kap:sicherheitstechnischeAnforderungen}) beschrieben werden. Dies ist dem Umstand geschuldet, dass alle relevanten Sicherheitsvorkehrungen durch die \ac{SVI} und ihre Dienstleisterinnen implementiert sind. Deren Implementierung ist eine zwingende Anforderung der \ac{SVI}, bevor ein System (Anwendungen, Server o.\,Ä.) produktiv geht. Die abschließende Entwicklung eines Sicherheitskonzeptes muss vor dem Produktivgang der \textsc{OpenShift}-Anwendung nochmals überprüft und in Zusammenarbeit mit den IT-Dienstleisterinnen der \ac{SVI} verbessert werden. Kritisch zu betrachten ist, dass der Soll-Ist-Vergleich keine tiefgreifende Informationen generiert hat. 

\paragraph{Einschränkungen durch den Labor-Betrieb} Der Labor-Betrieb der \textsc{OpenShift}-Umgebung schränkt die Risikoanalyse ein, da in einem abgeschlossenen Labor in der \ac{SVI} die Anwendungs-spezifischen Sicherheitsvorkehrungen nicht implementiert werden, da dieses Labor keinen Zugang zur produktiven \ac{AWL} besitzt. Netzwerk-technisch wie auch physikalisch sind Labor-Umgebungen vollständig von der \ac{AWL} getrennt. Aus diesem Grund genügt der Soll-Ist-Vergleich der Labor-Umgebung und muss vergleichbar für die produktiven Systeme ausgeführt werden.

\paragraph{Ausblick} Falls die Geschäftsführung sich entscheidet, auch die Bestandssysteme in Container-Anwendungen zu verlagern, erhöhen sich die sicherheitsrelevanten Anforderungen, da dann Gesundheitsdaten der Kunden der \ac{SV} in den Containern verarbeitet werden. Dadurch müssen die Sicherheitsvorkehrungen ausgeweitet werden, um die \enquote{Cloud}-Plattform \textsc{OpenShift} weiter zu schützen. Bei produktiven Systemen reicht das hier dargestellte Vorgehen nicht aus: Es müssten bei allen Komponenten eigene Risikoanalysen durchgeführt werden und die Verwendung der \ac{BSI}-Bausteine wäre nicht mehr ausreichend. In der produktiven \ac{AWL} werden sensible Daten verwaltet, die erhöhte Sicherheitsvorkehrungen nach der ISO-Norm 27001\autocite[vgl.][]{dindeutsches_institut_fur_normung_informationstechnik_2020} und der Empfehlung des \ac{BSI} im IT-Grundschutz\autocite[vgl.][]{bundesamt_fur_sicherheit_in_der_informationstechnik_bsi_it-grundschutz-kompendium_2020} benötigen. Eine weitere Risikobetrachtung durch die Teams der IT-\enquote{Compliance} der \ac{SVI} und der Riskobewertungsabteilung der \ac{SV} muss durchgeführt werden. 

\chapter{Epilog} \label{kritischeBetrachtung}
\paragraph{Zusammenfassung der Erkenntnisse}
\paragraph{Fazit}
\paragraph{Ausblick}

%-----------------
\clearpage
\pagenumbering{Roman}
\setcounter{page}{12} %TODO: schauen, ob die "9" passt.

%	Literaturverzeichnis
\printbibliography[title=Literaturverzeichnis]
\cleardoublepage

% Der Anhang beginnt hier - jedes Kapitel wird alphabetisch aufgezählt. (Anhang A, B usw.)
\appendix
\ihead{\appendixname~\thechapter} % Neue Header-Definition

% appendix.tex einziehen
\addcontentsline{toc}{chapter}{Anhang} %sorgt für eintrag ins inhaltsverzeichnis
\chapter{Ergänzungen zur Forschungsfrage eins} \label{appendixFF1}
In diesem Teil des Anhangs sind Ergänzungen zur Forschungsfrage eins des Kapitels \vref{ff1} beschrieben.

\section{Anforderungsdokument}\label{appendixAnforderung}

Ein Anforderungskatalog hat bestimmte Anforderungen, die an den Prozess gestellt werden. Neben der Forderung nach Einhaltung der Qualitätskriterien, definiert nach dem ISO-Standard 9000/9001, sind noch folgende Forderungen in der Literatur beschrieben: \autocite[sig.][S.\,34]{partsch_requirements-engineering_2010}

\begin{itemize}
	\item vollständig (inhaltlich – d.\,h., alle Anforderungen sind erfasst –, formal, Norm-konform)
	\item konsistent (keine Widersprüche zwischen den Bestandteilen des Dokuments,
	insbesondere keine Konflikte zwischen verschiedenen Anforderungen)
	\item lokal änderbar (Änderungen an einer Stelle sollten keine Einflüsse auf Konsistenz und Vollständigkeit des Gesamtdokuments haben)
	\item verfolgbar (ursprüngliche Stakeholderwünsche und Zusammenhänge zwischen
	Anforderungen sind leicht zu finden)
	\item klar strukturiert
	\item umfangsmäßig angemessen
	\item sortierbar/projizierbar (nach verschiedenen Kriterien, für verschiedene Stakeholder).
\end{itemize}

Die folgende Aufzählung beschreibt eine Vorlage für das Anforderungsdokument nach Quelle: Sie nutzt die Hilfsmittelsammlung \enquote{Volere}. Diese bietet im Themenbereich \enquote{requirements engineering} kostenpflichtig Dokumentenvorlagen an. Die beiden Bekanntesten sind die hier gezeigte \enquote{Volere Requirements Specification Template} und das kostenlose \enquote{Volere Atomic Requirement Template}, das umgangssprachlich \enquote{Snow Card} genannt wird. Die \enquote{Snow Card} (\vref{abb:volereSnowCard}) ist eine Karteikarte, die benutzt wird, um eine vollständige Aufnahme aller Informationen einer einzelnen Anforderung zu gewährleisten.\autocite[vgl.][]{VolereSnowCard} 

\begin{figure}[H]
	\centering
	\includegraphics[scale=0.6]{img/snowcard.pdf}
	\caption{Volere Snow Card}
	{\footnotesize Quelle: \cite{VolereSnowCard}}
	\label{abb:volereSnowCard}
	%		{\scriptsize \textit{Alle Rechte, einschließlich der Vervielfältigung, Veröffentlichung, Bearbeitung und Übersetzung bleiben der SV Informatik GmbH vorbehalten.}}
\end{figure}

Die folgende Liste wurde in Anlehnung an die Quelle \cite{VolereRequirmentsSpecTemplate} erstellt.

\begin{minipage}{\linewidth}
	\begin{itemize}\label{abb:volereReqSpec}
		\item Projekt-Treiber
		\begin{enumerate}
			\item Zweck des Projekts
			\item Auftraggeber, Kunde und andere Stakeholder
			\item Nutzer des Produkts
		\end{enumerate}
		\item Projekt-Randbedingungen
		\begin{enumerate}
			\item Einschränkungen
			\item Namenskonventionen und Definitionen
			\item Relevante Fakten und Annahmen
		\end{enumerate}
		\item Funktionale Anforderungen
		\begin{enumerate}
			\item Arbeitsrahmen
			\item Systemgrenzen
			\item Funktionale und Daten-Anforderungen
		\end{enumerate}
		\item Nicht-funktionale Anforderungen
		\begin{enumerate}
			\item Look-and-Feel-Anforderungen
			\item Usability-Anforderungen
			\item Performanz-Anforderungen
			\item Operationale und Umfeld-Anforderungen
			\item Wartungs- und Unterstützungsanforderungen
			\item Sicherheitsanforderungen
			\item Kulturelle und politische Anforderungen
			\item Rechtliche Anforderungen
		\end{enumerate}
		\item Projekt-Aspekte
		\begin{enumerate}
			\item Offene Punkte
			\item Standardlösungen
			\item Neu aufgetretene Probleme
			\item Installationsaufgaben
			\item Migrationstätigkeiten
			\item Risiken
			\item Kosten
			\item Nutzerdokumentation
			\item Zurückgestellte Anforderungen
			\item Lösungsideen
		\end{enumerate}
	\end{itemize}
\end{minipage}

\clearpage
\begin{figure}[H]
	\centering
	\includegraphics[scale=0.50, angle=90]{img/cloudreq.pdf}
	\caption{Ebenen der \enquote{Cloud}-Anforderungsanalyse}
	{\footnotesize Quelle: in Anlehnung an \cite{rimal_architectural_2011}}
	\label{abb:cloudreq}
	%		{\scriptsize \textit{Alle Rechte, einschließlich der Vervielfältigung, Veröffentlichung, Bearbeitung und Übersetzung bleiben der SV Informatik GmbH vorbehalten.}}
\end{figure}

\clearpage
\section{Statistiken zum Themengebiet \ac{Cloud-C}}

\begin{figure}[H]
	\centering
	\includegraphics[scale=0.43]{img/statistic_id150979_marktanteile-der-fuehrenden-unternehmen-im-bereich-cloud-computing-weltweit-2019.pdf}
	\caption{Marktanteile der führenden Unternehmen am Umsatz im Bereich \enquote{Cloud-Computing} weltweit von Juli 2018 bis Juni 2019}
	{\footnotesize Quelle: \cite{itcandor_cloud_2019}}
	\label{abb:marktanteileCC19}
	%		{\scriptsize \textit{Alle Rechte, einschließlich der Vervielfältigung, Veröffentlichung, Bearbeitung und Übersetzung bleiben der SV Informatik GmbH vorbehalten.}}
\end{figure}

\begin{figure}[H]
	\centering
	\includegraphics[scale=0.43]{img/statistic_id195760_prognose-zum-umsatz-mit-cloud-computing-weltweit-bis-2022.pdf}
	\caption{Umsatz mit \enquote{Cloud-Computing} weltweit von 2009 bis 2018 und Prognose bis 2022 }
	{\footnotesize Quelle: \cite{gartner_cloud_2019}}
	\label{abb:umsatzprognoseCC}
	%		{\scriptsize \textit{Alle Rechte, einschließlich der Vervielfältigung, Veröffentlichung, Bearbeitung und Übersetzung bleiben der SV Informatik GmbH vorbehalten.}}
\end{figure}

\section{Ergänzungen zum Kapitel Container, Containerisierung und Orchestrierung}

\begin{figure}[H]
	\centering
	\includegraphics[scale=0.45]{img/containerImageArch.pdf}
	\caption{Architektur des Container-\enquote{Images}}
	{\footnotesize Quelle: in Anlehnung an \cite{pahl_containerization_2015}}
	\label{abb:containerArch}
	%		{\scriptsize \textit{Alle Rechte, einschließlich der Vervielfältigung, Veröffentlichung, Bearbeitung und Übersetzung bleiben der SV Informatik GmbH vorbehalten.}}
\end{figure}

Hierbei ist zu beachten, dass das orange Gefärbte die Funktionalitäten der \textsc{Docker}-\enquote{Engine} und das blau Gefärbte die möglichen Bestandteile eines \textsc{Docker}-\enquote{Images} darstellen.


\begin{figure}[H]
	\centering
	\includegraphics[scale=0.46]{img/k8sArch.pdf}
	\caption{Überblick über eine \textsc{Kubernetes}-Architektur}
	{\footnotesize Quelle: in Anlehnung an \cite[][S.\,23]{luksa_kubernetes_2018}}
	\label{abb:k8sArch}
	%		{\scriptsize \textit{Alle Rechte, einschließlich der Vervielfältigung, Veröffentlichung, Bearbeitung und Übersetzung bleiben der SV Informatik GmbH vorbehalten.}}
\end{figure}

\begin{figure}[H]
	\centering
	\includegraphics[scale=0.46]{img/k8sArchInteraktion.pdf}
	\caption{Architektur der \enquote{\ac{K8s}}-Interaktion}
	{\footnotesize Quelle: in Anlehnung an \cite[][S.\,15]{caban_architecting_2019}}
	\label{abb:k8sArchInteraktion}
	%		{\scriptsize \textit{Alle Rechte, einschließlich der Vervielfältigung, Veröffentlichung, Bearbeitung und Übersetzung bleiben der SV Informatik GmbH vorbehalten.}}
\end{figure}


\section{Anforderungskatalog zur Forschungsfrage eins}
Die Klassifizierung beschreibt die Wichtigkeit (gering, mittel, hoch) im Kontext des
zu implementierenden Prozesses und die Nummer dient der besseren Identifizierbarkeit der jeweiligen Anforderung.
\begin{table*}[h!]
	\centering
	\ra{1.3} %more space beetween rules
	
	\begin{tabular}{@{}llp{9.0cm}@{}}\toprule[1.5pt]
		
		\textbf{Nummer} & \textbf{Klassifizierung} & \textbf{Anforderung} \\ \midrule
		% below rules with content
		
		$A_{1}$ & hoch & Das System muss die Möglichkeit bieten, über eine korrekte Konfigurationsdatei, alle Komponenten der Anwendung zu verteilen.           \\
		$A_{2}$ & mittel & Das System soll die Möglichkeit bieten über eine standardisierte Übergabedatei alle Umgebungsvariablen des Containers zu definieren. \\
		$A_{3}$ & hoch & Das System muss die Möglichkeit bieten, über eine Validierung, konsistente Komponenten zu verteilen.\\
		$A_{4}$ & hoch & Das System muss die Möglichkeit bieten unabhängig von dem Inhalt des Containers diesen zu verteilen.\\
		$A_{5}$ & hoch & Das System muss fähig sein mit \textsc{OpenShift} über das \ac{API} zu kommunizieren. \\
		$A_{6}$ & mittel & Das System soll fähig sein das \textsc{OpenShift}-Cluster zu jedem Zeitpunkt mit einem konsistenten Zustand zu verlassen. \\
		$A_{7}$ & niedrig & Das System wird die Möglichkeit bieten die Verteilung von Komponenten zu unterbrechen. \\
		$A_{8}$ & hoch & Das System muss die Möglichkeit bieten, über die Verteilungssoftware \textsc{Ara}, die Verteilung der Komponenten durchzuführen. \\
		$A_{9}$ & hoch & Das System muss die Möglichkeit bieten automatisiert alle notwendigen Komponenten über die Konfigurationsdatei zu erkennen und zu verteilen. \\
		$A_{10}$ & hoch & Das System muss den gesetzlichen sowie sicherheitstechnischen Vorgaben entsprechen.\\
		
		\bottomrule[1.5pt]
	\end{tabular}
	
	\caption{Anforderungskatalog zum zu implementierenden Prozess}
	\label{tab:anforderungslisteFF1}
	
\end{table*}

\clearpage

\section{Quellcode zum Testen der Webseiten-Verfügbarkeit}
% LISTING
\lstinputlisting[language=sh, caption={Funktion zum Test der Verfügbarkeit einer Webseite}, label=lst:testConnection]{resources/listings/testConnection.sh}

\clearpage

\section{Entwurf einer Konfigurationsdatei durch die Entwicklerinnen}
\lstinputlisting[language=yaml, caption={Entwurf einer \enquote{Deployment}-Datei für \textsc{Camunda}}, label=lst:entwurfCamunda]{resources/listings/camundaReadyToBePublished.yaml}

Dabei stellen alle Zeichenketten mit dem Muster \enquote{\textit{\$Bezeichner\_}} ein Platzhalter dar. Aus Gründen des Schutzes der Infrastruktur der \ac{SVI}/\ac{SV} ist es nicht gestattet die wirklichen Endpunkte der Webservices im Klartext abzubilden -- sie unterliegen dem Betriebsgeheimnis.

\clearpage
\section{Aufbau der zulässigen Wortfolgen der Übergabedatei}

\begin{lstlisting}[language=html, caption={\ac{BNF} der Übergabedatei}, label=lst:bnfDatei, mathescape=true]
$\langle$Zeile$\rangle$ ::= $\langle$Schluessel$\rangle$ $\langle$Zuweisung$\rangle$  $\langle$Wert$\rangle$ $\langle$EOL$\rangle$ | $\langle$Gruppe$\rangle$ $\langle$EOL$\rangle$ | $\lambda$ $\langle$EOL$\rangle$
$\langle$Schluessel$\rangle$ ::= $\langle$Grossbuchstaben$\rangle$ | $\langle$Grossbuchstaben$\rangle$ $\langle$Schluessel$\rangle$
$\langle$Zuweisung$\rangle$ ::= "="
$\langle$Wert$\rangle$ ::= $\langle$Zeichenkette$\rangle$
$\langle$Gruppe$\rangle$ ::= "$[$" $\langle$Zeichenfolge$\rangle$ "$]$"
$\langle$Zeichenfolge$\rangle$ ::= $\langle$Grossbuchstaben$\rangle$ | $\langle$Grossbuchstaben$\rangle$ $\langle$Zeichenfolge$\rangle$
$\langle$Zeichenkette$\rangle$ ::= $\lambda$ | $\langle$Zeichen$\rangle$ | $\langle$Zeichen$\rangle$ $\langle$Zeichenkette$\rangle$
$\langle$Grossbuchstaben$\rangle$ ::= "A" | "B" | "C" | "D" | "E" | "F" | "G" | "H" | "I" | "J" | "K" | "L" | "M" | "N" | "O" | "P" | "Q" | "R" | "S" | "T" | "U" | "V" | "W" | "X" | "Y" | "Z" | 
$\langle$Zeichen$\rangle$ ::= "A" | "B" | "C" | "D" | "E" | "F" | "G" | "H" | "I" | "J" | "K" | "L" | "M" | "N" | "O" | "P" | "Q" | "R" | "S" | "T" | "U" | "V" | "W" | "X" | "Y" | "Z" | "_" | "/" | "-" | ":" | "." | "a" | "b" | "c" | "d" | "e" | "f" | "g" | "h" | "i" | "j" | "k" | "l" | "m" | "n" | "o" | "p" | "q" | "r" | "s" | "t" | "u" | "v" | "w" | "x" | "y" | "z"
\end{lstlisting}
\clearpage

\section{Grundgerüst der Konfigurationsdatei}
\lstinputlisting[language=yaml, caption={Grundgerüst der Konfigurationsdatei}, label={lst:grundConfig}]{resources/listings/exampleDC.yaml}
In Anlehnung an \cite[][Application\,$\rightarrow$\,Deployments]{red_hat_inc_okd_2019}.

\par

\lstinputlisting[language=sh, caption={Beispielumsetzung der \ac{BNF} der Übergabedatei}, label={lst:bspUbergabe}]{resources/listings/beispielUbergabe.txt} 

%\lstinputlisting[language=python, caption={Beispielumsetzung des Algorithmus \vref{algo:configGenerierung}}, label={lst:bspUmsetzungAlgoGen}]{resources/listings/bspUmsetzungAlgoGen.py}


\chapter{Ergänzungen zur Forschungsfrage zwei} \label{appendixFF2}
In diesem Teil des Anhangs sind Ergänzungen zur Forschungsfrage zwei des Kapitels \vref{ff2} beschrieben.

\section{Entscheidung über die Notwendigkeit eines \enquote{Business Case}}

\begin{figure}[H]
	\centering
	\includegraphics[scale=0.48]{img/entscheidungBC.pdf}
	\caption{Notwendigkeit eines \enquote{Business Case}}
	{\footnotesize Quelle: in Anlehnung an \cite[][S.\,29]{brugger_it_2009}}
	\label{abb:entscheidungBC}
	%		{\scriptsize \textit{Alle Rechte, einschließlich der Vervielfältigung, Veröffentlichung, Bearbeitung und Übersetzung bleiben der SV Informatik GmbH vorbehalten.}}
\end{figure}

\section{Vor- und Nachteile der internen beziehungsweise externen Erstellung eines \enquote{Business Case}}\label{appendixVorNachteileErstellungBC}

\begin{table*}[h!]
	\centering
	\ra{1.3} %more space beetween rules
	
	\begin{tabular}{@{}ll@{}}\toprule[1.5pt]
		
		\textbf{Vorteile} & \textbf{Nachteile} \\ \midrule
		
		% below rules with content
		Neutralität & Kein Wissenstransfer \\
		Verfügbarkeit & Kosten, egal ob der \enquote{Business Case} rentabel ist\\
		Effiziente Erstellung & Hohe Kosten im Vergleich zur internen Erstellung \\
		Glaubwürdigkeit & Abhängigkeit \\
		Qualität & Verlust der firmeninternen Standards \\
		Innovation & \\
		Schlichtung & \\
		
		\bottomrule[1.5pt]
	\end{tabular}
	
	\caption{Überblick über die Vor- und Nachteile der externen Erstellung eines \enquote{Business Case}}
	{\footnotesize Quelle: in Anlehnung an \cite[][S.\,34]{brugger_it_2009}}
	\label{tab:externVorNachteile}
	
\end{table*}

\begin{table*}[h!]
	\centering
	\ra{1.3} %more space beetween rules
	
	\begin{tabular}{@{}ll@{}}\toprule[1.5pt]
		
		\textbf{Vorteile} & \textbf{Nachteile} \\ \midrule
		
		% below rules with content
		Wissensaufbau & Verfügbarkeit \\
		Qualität & Effizienzverlust bei rein technischen/operativen Mitarbeitern \\
		\enquote{Teamwork} & Glaubwürdigkeit \\
		Standardisierung & Qualitätskontrolle durch \enquote{Controlling}-Division \\
		
		\bottomrule[1.5pt]
	\end{tabular}
	
	\caption{Überblick über die Vor- und Nachteile der internen Erstellung eines \enquote{Business Case}}
	{\footnotesize Quelle: in Anlehnung an \cite[][S.\,34]{brugger_it_2009}}
	\label{tab:internVorNachteile}
	
\end{table*}

\begin{figure}[H]
	\centering
	\includegraphics[scale=0.48]{img/chronoBC.pdf}
	\caption{Chronologische Abfolge der Entwicklungsphase eines \enquote{Business Case}}
	{\footnotesize Quelle: in Anlehnung an \cite[][]{herman_is_2009}}
	\label{abb:entwicklungBC}
	%		{\scriptsize \textit{Alle Rechte, einschließlich der Vervielfältigung, Veröffentlichung, Bearbeitung und Übersetzung bleiben der SV Informatik GmbH vorbehalten.}}
\end{figure}

\chapter{Ergänzungen zur Forschungsfrage drei} \label{appendixFF3}
In diesem Teil des Anhangs sind Ergänzungen zur Forschungsfrage drei des Kapitels \vref{ff3} beschrieben.

\section{\enquote{Plan-Do-Check-Act}-Regelkreis}

\begin{figure}[H]
	\centering
	\includegraphics[scale=0.51]{img/planDoCheckAct.pdf}
	\caption{\enquote{Plan-Do-Check-Act}-Regelkreis}
	{\footnotesize Quelle: in Anlehnung an \cite[][S.\,12]{kersten_it-sicherheitsmanagement_2020}}
	\label{abb:planDoCheckAct}
	%		{\scriptsize \textit{Alle Rechte, einschließlich der Vervielfältigung, Veröffentlichung, Bearbeitung und Übersetzung bleiben der SV Informatik GmbH vorbehalten.}}
\end{figure}

\clearpage % neue Seite
\section{Checkliste zur Vorbereitung der \acs{ISMS}-Einführung}
\begin{table*}[h!]
	\centering
	\ra{1.3} %more space beetween rules
	
	\begin{tabular}{@{}lp{10cm}l@{}}\toprule[1.5pt]
		
		\textbf{Aktion} & \textbf{Gegenstand} & \textbf{Erfüllt?} \\ \midrule
		% below rules with content
		
		1 & Sind die Normen (27000, 27001, 27002) in aktueller elektronischer
		Form vorhanden? & $\square$ \\
		2 & Sind die Vorteile und der Nutzen eines ISMS erläutert worden? & $\square$ \\
		3 & Ist ein Grob-Abgleich mit ISO 27001 erfolgt? (Ziel: erste
		Aufwandsabschätzung) & $\square$ \\
		4 & Ist eine Entscheidung zur Orientierung an der ISO 27001
		getroffen worden? & $\square$ \\
		5 & Denken wir in Management-Systemen? Existieren schon andere
		Management-Systeme? & $\square$ \\
		6 & Ist der Begriff ISMS eingeführt? & $\square$ \\
		7 & Denken wir in Geschäftsprozessen und informationstechnischen Anwendungen? & $\square$ \\
		8 & Ist der Anwendungsbereich des ISMS (Scope) zumindest grob
		skizziert? & $\square$ \\
		9 & Sind zumindest die Top Level Assets und deren Asset/Risk
		Owner erfasst worden? & $\square$ \\
		10 & Wurden – zumindest grob – Sicherheitsziele für diese Assets
		festgelegt? & $\square$ \\
				
		\bottomrule[1.5pt]
	\end{tabular}
	
	\caption{Checkliste zur Vorbereitung der \ac{ISMS}-Einführung}
	{\footnotesize{Quelle: in Anlehnung an \cite[][S.\,15]{kersten_it-sicherheitsmanagement_2020}}}
	\label{tab:checklisteVorbereitungISMS}
	
\end{table*}

\clearpage
\section{Schichtenmodell des IT-Grundschutzes}
\begin{figure}[H]
	\centering
	\includegraphics[scale=0.45]{img/bsiSchichtenmodell.pdf}
	\caption{Schichtenmodell des IT-Grundschutzes}
	{\footnotesize Quelle: in Anlehnung an \cite[][S.\,9]{bundesamt_fur_sicherheit_in_der_informationstechnik_bsi_it-grundschutz-kompendium_2020}}
	\label{abb:BSISchichtenmodell}
	%		{\scriptsize \textit{Alle Rechte, einschließlich der Vervielfältigung, Veröffentlichung, Bearbeitung und Übersetzung bleiben der SV Informatik GmbH vorbehalten.}}
\end{figure}

"Die Prozessbausteine, die in der Regel für sämtliche oder große Teile eines Informationsverbunds gleichermaßen
gelten, unterteilen sich in die folgenden Schichten, die wiederum aus weiteren Teilschichten bestehen können.
\begin{itemize}
	\item Die Schicht ISMS enthält als Grundlage für alle weiteren Aktivitäten im Sicherheitsprozess den Baustein Sicherheitsmanagement.
	\item Die Schicht ORP befasst sich mit organisatorischen und personellen Sicherheitsaspekten. In diese Schicht fallen beispielsweise die Bausteine Organisation und Personal.
	\item Die Schicht CON enthält Bausteine, die sich mit Konzepten und Vorgehensweisen befassen. Typische Bausteine
	der Schicht CON sind unter anderem Kryptokonzept und Datenschutz.
	\item Die Schicht OPS umfasst alle Sicherheitsaspekte betrieblicher Art. Insbesondere sind dies die Sicherheitsaspekte des operativen IT-Betriebs, sowohl bei einem Betrieb im Haus, als auch bei einem IT-Betrieb, der in Teilen oder komplett durch Dritte betrieben wird. Ebenso enthält er die Sicherheitsaspekte, die bei einem IT-Betrieb für Dritte zu beachten sind. Beispiele für die Schicht OPS sind die Bausteine Schutz vor Schadprogrammen und Outsourcing für Kunden.
	\item In der Schicht DER finden sich alle Bausteine, die für die Überprüfung der umgesetzten Sicherheitsmaßnahmen, die Detektion von Sicherheitsvorfällen sowie die geeigneten Reaktionen darauf relevant sind. Typische Bausteine der Schicht DER sind Behandlung von Sicherheitsvorfällen und Vorsorge für IT-Forensik.
\end{itemize}
Neben den Prozess-Bausteinen beinhaltet das IT-Grundschutz-Kompendium auch System-Bausteine. Diese werden
in der Regel auf einzelne Zielobjekte oder Gruppen von Zielobjekten angewendet. Die System-Bausteine unterteilen sich in die folgenden Schichten. Ähnlich wie die Prozess-Bausteine können auch die System-Bausteine aus weiteren Teilschichten bestehen.

\begin{itemize}
	\item Die Schicht APP beschäftigt sich mit der Absicherung von Anwendungen und Diensten, unter anderem in den
	Bereichen Kommunikation, Verzeichnisdienste, netzbasierte Dienste sowie Business- und Client-Anwendungen.
	Typische Bausteine der Schicht APP sind Allgemeine Groupware, Office-Produkte, Webserver und Relationale
	Datenbanksysteme.
	\item Die Schicht SYS betrifft die einzelnen IT-Systeme des Informationsverbunds, die ggf. in Gruppen zusammengefasst wurden. Hier werden die Sicherheitsaspekte von Servern, Desktop-Systemen, Mobile Devices und sonstigen IT-Systemen wie Druckern und TK-Anlagen behandelt. Zur Schicht SYS gehören beispielsweise Bausteine zu
	konkreten Betriebssystemen, Allgemeine Smartphones und Tablets sowie Drucker, Kopierer und Multifunktionsgeräte.
	\item Die Schicht IND befasst sich mit Sicherheitsaspekten industrieller IT. In diese Schicht fallen beispielsweise die Bausteine Betriebs- und Steuerungstechnik, Allgemeine ICS-Komponente und Speicherprogrammierbare [sic!] Steuerung (SPS).
	\item Die Schicht NET betrachtet die Vernetzungsaspekte, die sich nicht primär auf bestimmte IT-Systeme, sondern
	auf die Netzverbindungen und die Kommunikation beziehen. Dazu gehören zum Beispiel die Bausteine NetzManagement, Firewall und WLAN-Betrieb.
	\item Die Schicht INF befasst sich mit den baulich-technischen Gegebenheiten, hier werden Aspekte der infrastrukturellen Sicherheit zusammengeführt. Dies betrifft unter anderem die Bausteine Allgemeines Gebäude und Rechenzentrum."\autocite[][S.\,23-24]{bundesamt_fur_sicherheit_in_der_informationstechnik_bsi_it-grundschutz-kompendium_2020}
\end{itemize}



\chapter{Ergänzungen zum Literaturverzeichnis}
In diesem Kapitel werden die PDF-Dokumente der zitierten Webseiten dargestellt. Dies dient der Nachvollziehbarkeit der Quellen. Diese sind Auszüge aus den relevanten Webseiten und ohne Formatierung der Seiten des Dokuments beigefügt. Die nachfolgende Liste zählt die Quellen der Reihenfolge nach auf:

\begin{enumerate}
	\item \cite{google_llc_google_2020}
	\item \cite[][Application\,$\rightarrow$\,Deployments]{red_hat_inc_okd_2019}
	\item \cite{google_ireland_limited_container_2020}
	\item \cite{red_hat_inc_cli_2020}
	\item \cite{sonatype_inc_nexus_2020}
	\item \cite{broadcom_inc_automic_2020}
	\item \cite{camunda_services_gmbh_workflow_2020}
	\item \cite{canonical_ltd_linux_2020}
	\item \cite{django_software_foundation_web_2020}
	\item \cite{google_llc_production-grade_2020}
	\item \cite{opensourceorg_open_2020}
	\item \cite{rupp_formulierungsregel_2020}
	\item \cite{the_people_of_the_gnupg_project_gnu_2020}
	
	
\end{enumerate}

\includepdf[pagecommand={\thispagestyle{myheadings}}, pages=1, scale=0.8]{reference-includes/google-llc-2020a}
\includepdf[pagecommand={\thispagestyle{myheadings}}, pages=1-3, scale=0.8]{reference-includes/red-hat-applications-deployment}
\includepdf[pagecommand={\thispagestyle{myheadings}}, pages=1-2, scale=0.8]{reference-includes/google-limited-2020}
\includepdf[pagecommand={\thispagestyle{myheadings}}, pages=1-2, scale=0.8]{reference-includes/red hat 2020a}
\includepdf[pagecommand={\thispagestyle{myheadings}}, pages=1, scale=0.8]{reference-includes/sonatype inc 2020}
\includepdf[pagecommand={\thispagestyle{myheadings}}, pages=1, scale=0.8]{reference-includes/broadcom Automic Automation}
\includepdf[pagecommand={\thispagestyle{myheadings}}, pages=1, scale=0.8]{reference-includes/camunda}
\includepdf[pagecommand={\thispagestyle{myheadings}}, pages=1, scale=0.8]{reference-includes/linux container}
\includepdf[pagecommand={\thispagestyle{myheadings}}, pages=1, scale=0.8]{reference-includes/django}
\includepdf[pagecommand={\thispagestyle{myheadings}}, pages=1, scale=0.8]{reference-includes/k8s}
\includepdf[pagecommand={\thispagestyle{myheadings}}, pages=1-3, scale=0.8]{reference-includes/opensource}
\includepdf[pagecommand={\thispagestyle{myheadings}}, pages=1, scale=0.8]{reference-includes/anforderungsschabloneRupp}
\includepdf[pagecommand={\thispagestyle{myheadings}}, pages=1, scale=0.8]{reference-includes/gnupg}




% Ehrenwörtliche Erklärung ewerkl.tex einziehen
% !TEX root =  master.tex

\clearpage
\chapter*{Ehrenwörtliche Erklärung}

% Wird die folgende Zeile auskommentiert, erscheint die ehrenwörtliche
% Erklärung im Inhaltsverzeichnis.
\addcontentsline{toc}{chapter}{Ehrenwörtliche Erklärung}

Ich versichere hiermit, dass ich die vorliegende Arbeit mit dem Thema: \textit{\DerTitelDerArbeit} selbstständig verfasst und keine anderen als die angegebenen Quellen und Hilfsmittel benutzt habe. Ich versichere zudem, dass die eingereichte elektronische Fassung mit der gedruckten Fassung übereinstimmt.

%\vspace{3cm}
%\noindent\rule{5cm}{.4pt}\hfill\rule{5cm}{.4pt}\par
%Ort, Datum \hfill \DerAutorDerArbeit

\vspace{3cm}
\noindent Mannheim, 08.05.2020 \hfill\rule{5cm}{.4pt}\par
\noindent\hfill \DerAutorDerArbeit



\end{document}
